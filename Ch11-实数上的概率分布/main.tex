\documentclass[UTF8, a4paper]{article}
\usepackage{ctex}
\usepackage{graphicx}
\usepackage[margin=2.5cm]{geometry}
\usepackage{subcaption}
\usepackage{amssymb}
\usepackage{amsthm}
\usepackage{amsmath}
\usepackage{enumerate}
\usepackage[backend=bibtex, style=alphabetic]{biblatex}
\usepackage{framed}
\usepackage{mathrsfs} 
\usepackage{xcolor}
\newtheorem{exercise}{Exercise \#11.}
\newtheorem*{proposition}{命题}
\newtheorem*{remark}{注}
\everymath{\displaystyle}

\addbibresource{my.bib}
\title{Chapter 11: 实数上的概率分布}
\author{}
\date{Latest Update: \today}
\begin{document}
\maketitle


\begin{framed}
\begin{exercise}
用\(\chi^2\)随机变量的密度证明\(\Gamma\left(\frac{1}{2}\right) = \sqrt{\pi}\).
\end{exercise}
\end{framed}

\begin{proof}
由于\(\chi^2 \sim \Gamma\left(\frac{1}{2}, 2\right)\), 密度函数为
$$
\frac{1}{2^{k / 2} \Gamma(k / 2)} x^{k / 2-1} e^{-x / 2}
$$
考察\(X \sim N(0, 1)\), 则令\(Y = X^2\)
当\(X < 0\)时, \(X = -\sqrt{Y}\); 当\(X \geq 0\)时, \(X = \sqrt{Y}\).
Jacobian 为\(\frac{1}{2\sqrt{y}}\), 从而\(Y\)的密度函数为
$$
f_Y(y) =2\cdot \frac{1}{2\sqrt{2\pi y}} e^{-y/2} \mathbb{I}_{(0, \infty)}(y) = \frac{1}{\sqrt{2} \Gamma(1/2)} y^{-1/2} e^{-y/2} \mathbb{I}_{(0,\infty)}.
$$
于是, \(\Gamma\left(\frac{1}{2}\right) = \sqrt{\pi}\).
\end{proof}


\begin{framed}
\begin{exercise}
设\(X\)是\([-1,1]\)上的均匀分布. 求出\(Y = X^k\)的密度, 其中\(k\)是正整数.
\end{exercise}
\end{framed}

\begin{proof}
由于\(X\)是\([-1,1]\)上的均匀分布, 密度函数为
$$
f_X(x) = \frac{1}{2} \mathbb{I}_{[-1,1]}(x).
$$
当\(k\)是奇数时, \(g(x) = x^k\)是一个一一变换, 
反函数\(h(y) = g^{-1}(y) = y^{1/k}\), 
\(h'(y) = \frac{1}{k} y^{1/k - 1}\),
从而\(Y\)的密度函数为
$$
f_Y(y) = f_X(h(y)) \left|h'(y)\right| = \frac{1}{2k}y^{\frac{1}{k} - 1} \mathbb{I}_{[-1,1]}(y).
$$

当\(k\)是偶数时, \(g(x) = x^k\)不是一个一一变换, 但是\(g(x) = x^k\)分别在\(I_1 = [-1,0), I_2 = [0,1]\)上是一个一一变换,
且\(g(I_1) = (0, 1], g(I_2) = [0, 1]\), 
反函数\(h_1(y) = g^{-1}(y) = -y^{1/k}\), \(h_2(y) = y^{1/k}\),
且导数为
$$
h_1'(y) = -\frac{1}{k} y^{1/k - 1}, h_2'(y) = \frac{1}{k} y^{1/k - 1},
$$
从而\(Y\)的密度函数为
$$
f_y(y) = f_X(h_1(y))|h_1'(y)| \mathbb{I}_{(0,1]}(y) + f_X(h_2(y))|h_2'(y)| \mathbb{I}_{[0,1]}(y) = \frac{1}{k}y^{\frac{1}{k} - 1} \mathbb{I}_{[0,1]}(y).
$$

\end{proof}


\begin{framed}
\begin{exercise}
设\(X\)的分布函数是\(F\). 求出\(Y = |X|\)的分布函数. 当\(X\)有一个连续的密度函数\(f_X\)时, 用\(f_X\)表达出\(Y\)的密度函数\(f_Y\).
\end{exercise}
\end{framed}

\begin{proof}
\(Y = |X|\)的分布函数为
$$
F_Y(y) = P(|X| \leq y) = P(-y \leq X \leq y) = F(y) - F(-y - 0).
$$
当\(X\)有一个连续的密度函数\(f_X\)时,\(F_Y(-y-0) = F_Y(-y)\), 于是\(F_Y(y) = F(y) - F(-y)\), 从而\(Y\)有一个密度函数\(f_Y\), 且
$$
f_Y(y) = F_Y'(y) = F'(y) - F'(-y) = f_X(y) + f_X(-y).
$$
\end{proof}



\begin{framed}
\begin{exercise}
设\(X\)服从参数为\(\left(\alpha, 1\right)\)的Cauchy分布.
设\(Y = \frac{a}{X}\), 其中\(a \neq 0\). 证明: \(Y\)还是一个Cauchy随机变量, 它的参数是\(\left(\frac{a\alpha}{1+\alpha^2}, \sqrt{\frac{|a|}{1+\alpha^2}}\right)\).
\end{exercise}
\end{framed}

\begin{proof}
按教材P44关于Cauchy分布的定义, Cauchy\((\alpha, \beta)\)的pdf为
$$
f(x) = \frac{1}{\beta \pi}\frac{1}{1+(x-\alpha)^2/\beta^2}.
$$
则\(X\)的密度函数是 
$$
f_X(x) = \frac{1}{\pi}\frac{1}{1+(x - \alpha)^2}.
$$
由于\(Y = g(X) = \frac{a}{X}\). 记\(S_0 = \{0\}, S_1 = (0, \infty), S_2 = (-\infty, 0)\),
从而有\(\mathbb{R} = S_0 + S_1 + S_2\), 且\(g\)在\(S_1\)和\(S_2\)上是单射, \(m_1(S_0) = 0\).

在\(S_1\)上, \(g\)的逆函数为\(g^{-1}(y) = \frac{a}{y}\), 从而\(g^{-1}\)的导数为\(\frac{-a}{y^2}\), 
在\(S_2\)上, \(g\)的逆函数为\(g^{-1}(y) = \frac{a}{y}\), 从而\(g^{-1}\)的导数为\(\frac{a}{y^2}\).
从而\(Y\)的密度函数为
$$
\begin{aligned}
    f_Y(y) =& \frac{1}{\pi}\frac{1}{1+\left(\frac{a}{y} - \alpha\right)^2}\frac{|a|}{y^2} \\
    % =& \frac{1}{\pi}\frac{a}{y^2+a^2 + \alpha^2 y^2 - 2a\alpha y} \\
    =& \frac{1}{\pi}\frac{|a|}{y^2(1+\alpha^2) + a^2 - 2a\alpha y} \\
    =& \frac{1}{\pi}\frac{|a|}{(1+\alpha^2)\left(y^2 -\frac{2a\alpha}{1+\alpha^2}y + \frac{a^2\alpha^2}{(1+\alpha^2)^2} \right) + a^2 - \frac{a^2\alpha^2}{1+\alpha^2}} \\
    =& \frac{1}{\pi}\frac{1}{\frac{(1+\alpha^2)}{a}\left(y -\frac{a\alpha}{1+\alpha^2}\right)^2 + \frac{|a|}{1+\alpha^2}} \\
    =& \frac{1}{\pi \cdot \frac{|a|}{1+\alpha^2}}\frac{1}{\frac{(1+\alpha^2)^2}{a^2}\left(y -\frac{a\alpha}{1+\alpha^2}\right)^2 + 1} \\
\end{aligned}.
$$
从而\(Y\)服从参数为\(\left(\frac{a\alpha}{1+\alpha^2}, {\frac{|a|}{1+\alpha^2}}\right)\)的Cauchy分布.
\end{proof}



\begin{framed}
\begin{exercise}
设\(X\)的密度函数是\(f_X\), 设\(Y = \frac{a}{X}\), 其中\(a \neq 0\). 
用\(f_X\)表示出\(Y\)的密度函数\(f_Y\).
\end{exercise}
\end{framed}

\begin{proof}
由于\(Y = \frac{a}{X}\), \(Y\)的分布函数为 
$$
\begin{aligned}
F_Y(y) &= P(Y \leq y) = P\left(\frac{a}{X} \leq y\right).
\end{aligned}
$$
若\(a>0, y>0\), 则
$$
\begin{aligned}
F_Y(y) &= P\left(X \geq \frac{a}{y}\right) + P(X < 0) = 1 - F_X\left(\frac{a}{y}\right) + F_X(0-0).
\end{aligned}
$$
若\(a>0, y\leq 0\), 则
$$
\begin{aligned}
F_Y(y) &= P\left(\frac{a}{y} \leq X < 0\right) = F_X(0-0) - F_X\left(\frac{a}{y} - 0\right).
\end{aligned}
$$
此时, \(Y\)的密度函数为
$$
f_Y(y) = F_Y'(y) = f_X\left(\frac{a}{y}\right) \frac{a}{y^2} \mathbb{I}_{(0, \infty)}(y) + f_X\left(\frac{a}{y}\right) \frac{a}{y^2} \mathbb{I}_{(-\infty, 0]}(y) = \frac{a}{y^2} f_X\left(\frac{a}{y}\right).
$$

若\(a < 0, y > 0\), 则
$$
\begin{aligned}
F_Y(y) &= P\left(X \leq \frac{a}{y}\right) + P(X > 0) = F_X\left(\frac{a}{y}\right) + 1 - F_X(0-0).
\end{aligned}
$$
若\(a < 0, y \leq 0\), 则
$$
\begin{aligned}
F_Y(y) &= P\left(0 < X \leq \frac{a}{y}\right) = F_X\left(\frac{a}{y}\right) - F_X(0).
\end{aligned}
$$
此时, \(Y\)的密度函数为
$$
f_Y(y) = F_Y'(y) = f_X\left(\frac{a}{y}\right) \frac{-0a}{y^2} \mathbb{I}_{(0, \infty)}(y) + f_X\left(\frac{a}{y}\right) \frac{-a}{y^2} \mathbb{I}_{(-\infty, 0]}(y) = \frac{-a}{y^2} f_X\left(\frac{a}{y}\right).
$$

综上, 
$$
f_Y(y) = \frac{|a|}{y^2} f_X\left(\frac{a}{y}\right).
$$
\end{proof}


\begin{framed}
\begin{exercise}
设\(X\)是\((-\pi, \pi)\)上的均匀分布, 令\(Y = \sin(X+ \theta)\). 
证明: \(Y\)的密度函数是\(f_Y(y) = \frac{1}{\pi\sqrt{1-y^2}}\), 其中\(-1 < y < 1\).
\end{exercise}
\end{framed}

\begin{proof}
由于\(X\)是\((-\pi, \pi)\)上的均匀分布, 密度函数为
$$
f_X(x) = \frac{1}{2\pi} \mathbb{I}_{(-\pi, \pi)}(x).
$$
先证明: \(P(\sin(X + \theta) \leq y) = P(\sin(X) \leq y), \forall y, \theta\).
事实上, 由于考虑的\(Z = X+\theta\)取值范围也是一个长度为\(2\pi\)的区间, 由于\(\sin\)是周期函数, 
从而积分区域长度一致, 从而\(P(\sin(X + \theta) \leq y) = P(\sin(X) \leq y)\).

接下来, 考察\(Y = \sin(X)\), 
令\(g(x) = \sin(x)\), 
于是\(g(x)\)分别在\(I_1 = (-\pi, -\pi/2), I_2 = (-\pi/2, \pi/2), I_3 = (\pi/2, \pi)\)上是一个一一变换, \(I_0 = \{-\pi/2, \pi/2\}\)是一个零测集,
且\(g(I_1) = (-1, 0), g(I_2) = [-1, 1], g(I_3) = (0, 1)\).
反函数\(h_1(y) = \arcsin(y) - \pi, h_2(y) = \arcsin(y), h_3(y) = \arcsin(y) + \pi\),
导数分别为
$$
h_1'(y) = h_2'(y) = h_3'(y) = \frac{1}{\sqrt{1-y^2}}.
$$
从而\(Y\)的密度函数为
$$
\begin{aligned}
f_Y(y) &= f_X(h_1(y))|h_1'(y)| \mathbb{I}_{(-1, 0)}(y) + f_X(h_2(y))|h_2'(y)| \mathbb{I}_{(-1, 1)}(y) + f_X(h_3(y))|h_3'(y)| \mathbb{I}_{(0, 1)}(y) \\
&= \frac{1}{2\pi} \frac{1}{\sqrt{1-y^2}} \mathbb{I}_{(-1, 0)}(y) + \frac{1}{2\pi} \frac{1}{\sqrt{1-y^2}} \mathbb{I}_{(-1, 1)}(y) + \frac{1}{2\pi} \frac{1}{\sqrt{1-y^2}} \mathbb{I}_{(0, 1)}(y) \\
&= \frac{1}{\pi\sqrt{1-y^2}}.
\end{aligned}
$$


% 显然, 当\(y \leq -1\)时, \(P(Y\leq y) = 0\), 当\(y \geq 1\)时, \(P(Y \leq y) = 1\). 
% 当\(-1 < y < 1\)时,
% $$
% P(Y \leq y) = P(\sin(X+\theta) \leq y) = \int_{\sin(x + \theta) \leq y} \frac{1}{2\pi} \mathbb{I}_{(-\pi, \pi)}(x) dx.
% $$
% 由于 
% $$
% \sin(x+ \theta) \leq y  \Leftrightarrow x+\theta \in [\arcsin(y) + 2k\pi, \pi - \arcsin(y) + 2k\pi], k \in \mathbb{Z},
% $$
\end{proof}


\begin{framed}
\begin{exercise}
设\(X\)有密度函数\(f_X\). 令\(Y = a\sin(X+\theta)\), 其中\(a > 0\).
证明:$$
f_Y(y)=\frac{1}{\sqrt{a^2-y^2}} \sum_{i=-\infty}^{\infty}\left[f_X\left\{h_i(y)\right\}+f_X\left\{k_i(y)\right\}\right] 1_{[-a, a]}(y)
$$
对于适当的函数\(h_i(y)\)和\(k_i(y)\).
\end{exercise}
\end{framed}

\begin{proof}
类似上面的讨论, 首先\(Y\)的分布函数与参数\(\theta\)无关, 从而可以取\(\theta = 0\).

\(Y = a\sin(X)\), 
令\(g(x) = a\sin(x)\), 则\(g(x)\)在\(I_k = (-\pi/2 + k\pi, \pi/2 + k\pi)\)上是一个一一变换, \(k \in \mathbb{Z}\), \(I_0 = \{\pi/2 + k\pi\}_{k\in\mathbb{Z}}\)是一个零测集,
且\(g(I_k) = [-a, a]\), \(k \in \mathbb{Z}\).
反函数, 当\(k = 2m\)时, \(\phi_k(y) = \arcsin(y/a) + k\pi\), 当\(k = 2m+1\)时, \(\phi_k(y) = \pi - \arcsin(y/a) + k\pi\), \(m \in \mathbb{Z}\),
导数分别为
$$
\phi_k'(y) = \frac{1}{\sqrt{a^2 - y^2}}.
$$
记\(h_i(y) = \phi_{2i}(y), k_i(y) = \phi_{2i+1}(y)\), 则\(Y\)的密度函数为
$$
\begin{aligned}
f_Y(y) &= \sum_{m \in \mathbb{Z}} \left[f_X\left\{\phi_{2m}(y)\right\} + f_X\left\{\phi_{2m+1}(y)\right\}\right] \frac{1}{\sqrt{a^2 - y^2}} 1_{[-a, a]}(y) \\
&= \frac{1}{\sqrt{a^2-y^2}} \sum_{i=-\infty}^{\infty}\left[f_X\left\{h_i(y)\right\}+f_X\left\{k_i(y)\right\}\right] 1_{[-a, a]}(y).
\end{aligned}
$$

\end{proof}



\begin{framed}
\begin{exercise}
设\(X\)是\((-\pi, \pi)\)上的均匀分布, 令\(Y = a\tan(X)\), 其中\(a > 0\). 求出\(Y\)的密度函数\(f_Y\).
\end{exercise}
\end{framed}

\begin{proof}
由于\(X\)是\((-\pi, \pi)\)上的均匀分布, 密度函数为
$$
f_X(x) = \frac{1}{2\pi} \mathbb{I}_{(-\pi, \pi)}(x).
$$

\(g(x) = a\tan(x)\)分别在\(I_1 = (-\pi, -\pi/2), I_2 = (-\pi/2, \pi/2), I_3 = (\pi/2, \pi)\)上是一个一一变换, 记\(I_0 = \{\pi/2, -\pi/2\}\), \((-\pi, \pi)= \cup_{i = 0}^3 I_i\), 且\(I_0\)是零测集.
且\(g(I_1) = (0, \infty), g(I_2) = \mathbb{R}^1, g(I_3) = (-\infty, 0)\).
于是, 分别单射区域上的反函数\(h_i\)分别为
$$
h_1(y) = \arctan\left(\frac{y}{a}\right) - \pi, h_2(y) = \arctan\left(\frac{y}{a}\right), h_3(y) = \arctan\left(\frac{y}{a}\right) + \pi.
$$
导数分别为
$$
h_1'(y) = h_2'(y) = h_3'(y) = \frac{1/a}{1 + \frac{y^2}{a^2}} = \frac{a}{a^2 + y^2}.
$$
于是, \(Y\)的密度函数为
$$
\begin{aligned}
f_Y(y) &= f_X(h_1(y))|h_1'(y)| \mathbb{I}_{g(I_1)}(y) + f_X(h_2(y))|h_2'(y)| \mathbb{I}_{g(I_2)}(y)+ f_X(h_3(y))|h_3'(y)|\mathbb{I}_{g(I_3)}(y) \\ 
&= \frac{1}{2\pi} \frac{a}{a^2 + y^2} \mathbb{I}(y>0) + \frac{1}{2\pi} \frac{a}{a^2 + y^2} + \frac{1}{2\pi} \frac{a}{a^2 + y^2} \mathbb{I}(y < 0) \\
&= \frac{1}{\pi} \frac{a}{a^2 + y^2}.
\end{aligned}
$$


\end{proof}


\begin{framed}
\begin{exercise}
设\(X\)有密度函数, 令
$$
Y=c e^{-\alpha X} 1_{\{X>0\}}, \quad(\alpha>0, c>0)
$$
用\(f_X\)表示\(Y\)的密度函数\(f_Y\).
\end{exercise}
\end{framed}

\begin{proof}
\(X\) 有密度函数, 则\(X\)的分布函数连续.
由于\(Y = c e^{-\alpha X} 1_{\{X>0\}}\), 当\(y > 0\)时, \(Y\)的分布函数为
$$
F_Y(y) = P(Y \leq y) = P\left(c e^{-\alpha X} 1_{\{X>0\}} \leq y\right) = P\left(X \geq -\frac{1}{\alpha}\ln\left(\frac{y}{c}\right), X > 0\right) .
$$
当\(y \geq c\)时, \(F_Y(y) = 1\); 当\(0 < y < c\)时, 
$$F_Y(y) = P\left(X \geq -\frac{1}{\alpha}\ln\left(\frac{y}{c}\right)\right) = 1 - F_X\left(-\frac{1}{\alpha}\ln\left(\frac{y}{c}\right)\right). $$
此时, \(Y\)的密度函数为
$$
f_Y(y) = F_Y'(y) = \frac{1}{\alpha y} f_X\left(-\frac{1}{\alpha}\ln\left(\frac{y}{c}\right)\right) \mathbb{I}_{(0, c)}(y).
$$
\end{proof}


\begin{framed}
\begin{exercise}
称一个密度函数\(f\)是{\it 对称的}, 如果对任意\(x\), 有\(f(x) = f(-x)\). (即, \(f\)是一个偶函数.)
若\(X\)与\(-X\)是同分布的, 则称随机变量\(X\)是{\it 对称的}.
假设\(X\)的密度函数是\(f\). 证明: \(X\)是对称的当且仅当\(f\)是对称的. 
在这种情况下, 它是否也允许存在一个非对称的密度函数?
\end{exercise}
\end{framed}
\begin{remark}
对称密度的例子是: \((-a,a)\)上的均匀分布; 参数为\((0, \sigma^2)\)的正态分布; 参数为\((0, \beta)\)的Cauchy分布, 参数为\((0, \beta)\)的双指数分布.
\end{remark}

\begin{proof}
\(\Rightarrow\): 若\(X\)是对称的, 则\(X\)与\(-X\)是同分布的,
$$
P(X \leq x) = P(-X \leq x) = P(X \geq -x) = 1 - P(X \leq -x).
$$
从而\(F(x) = 1 - F(-x)\), 从而\(f(x) = -f(-x)\), 即\(f\)是一个奇函数.

\(\Leftarrow\): 若\(f\)是对称的, 则\(f(x) = f(-x)\), 从而
$$
F(x) = \int_{-\infty}^x f(t)dt = \int_{-\infty}^x f(-t)dt = \int_{-\infty}^{-x} f(t)dt = 1 - F(-x).
$$
从而\(X\)与\(-X\)是同分布的.

由于在一个Lebesgue零测集上的函数值的改变不会影响积分, 从而可以修改\(f\)使之不对称, 但是\(X\)可以对称.
\end{proof}


\begin{framed}
\begin{exercise}
设\(X\)是正随机变量, 密度函数为\(f\). 令\(Y = \frac{1}{X+1}\), 求出\(Y\)的密度函数.
\end{exercise}
\end{framed}


\begin{proof}
由于\(X\)是正随机变量, 密度函数为\(f\). 
由于\(Y = \frac{1}{X+1} \in (0, 1)\), 则\(X = \frac{1}{Y} - 1\)是一个一一变换.
此时, Jacobian \(J = \frac{\partial X}{\partial Y} = -\frac{1}{Y^2} \not\equiv 0\), 从而\(Y\)的密度函数为
$$
\begin{aligned}
f_Y(y) &= f\left(\frac{1}{y} - 1\right) \left|-\frac{1}{y^2}\right| \\
&= \frac{1}{y^2} f\left(\frac{1}{y} - 1\right) \mathbb{I}_{(0,1)}(y).
\end{aligned}
$$
\end{proof}


\begin{framed}
\begin{exercise}
设\(X\)是参数为\((\mu, \sigma^2)\)的正态分布. 令\(Y = e^X\). 证明: \(Y\)服从对数正态分布.
\end{exercise}
\end{framed}

\begin{proof}
若\(Y\)服从对数正态分布, 则\(Y\)的密度函数为
$$
{\displaystyle f(x\mid \mu ,\sigma )={\frac {1}{x\sigma {\sqrt {2\pi }}}}e^{-(\ln x-\mu )^{2}/2\sigma ^{2}}}
$$
现在, 考察\(Y = e^X\), 则\(X = \ln Y\). 由于\(X\)是正态分布, 则\(X\)的密度函数为
$$
f_X(x) = \frac{1}{\sqrt{2\pi}\sigma} e^{-\frac{(x-\mu)^2}{2\sigma^2}}.
$$
从而\(Y\)的密度函数为
$$
\begin{aligned}
f_Y(y) &= f_X(\ln y) \left|\frac{1}{y}\right| \\
&= \frac{1}{y\sqrt{2\pi}\sigma} e^{-\frac{(\ln y - \mu)^2}{2\sigma^2}}.
\end{aligned}
$$

\end{proof}


\begin{framed}
\begin{exercise}
设\(X\)是分布函数为\(F\)的随机变量. 令\(Y = F(X)\). 证明: \(Y\)是\([0,1]\)上的均匀分布随机变量.
\end{exercise}
\end{framed}

\begin{proof}
定义
$$
F^{-1}(u) = \inf\{x: F(x) \geq u\},
$$
为\(F\)的反函数.
考察\(Y\)的分布函数:
$$
P(Y\leq y) = P(F(X) \leq y) = P(X \leq F^{-1}(y)) = F(F^{-1}(y)) = y.
$$
\end{proof}


\begin{framed}
\begin{exercise}
设\(F\)是一个连续的分布函数, 且\(F^{-1}\)存在. 设\(U\)是\((0,1)\)上的均匀分布随机变量. 证明: \(X = F^{-1}(U)\)的分布函数是\(F\).
\end{exercise}
\end{framed}

\begin{proof}
\(X\)的分布函数为
$$
F_X(x) = P(X \leq x) = P(F^{-1}(U) \leq x) = P(U \leq F(x)) = F(x).
$$
\end{proof}

\begin{framed}
\begin{exercise}
设\(F\)是一个连续的分布函数, \(U\)是\((0,1)\)上的均匀分布随机变量. 定义\(G(u) = \inf\{x: F(x) \geq u\}\). 证明: \(G(U)\)的分布函数是\(F\).
\end{exercise}
\end{framed}

\begin{proof}
设\(Y = G(U)\), 则\(Y\)的分布函数为
$$
F(y) = P(Y \leq y) = P(G(U) \leq y) = P(U \leq F(y)) = F(y).
$$
得证.
\end{proof}


\begin{framed}
\begin{exercise}
设\(Y = -\frac{1}{\lambda} \ln(U)\), 其中\(U\)是\((0,1)\)上的均匀分布随机变量. 证明: \(Y\)服从参数为\(\lambda\)的指数分布.
\end{exercise}
\end{framed}
\begin{remark}
这给出来一种模拟指数随机变量的方法.
\end{remark}


\begin{proof}
    服从参数为\(\lambda\)的指数分布的随机变量\(X\)的密度函数和分布函数分别为:
$$
\begin{aligned}
& p(x)=\lambda e^{-\lambda x}, x>0 \\
& F(x)=1-e^{-\lambda x}, x>0
\end{aligned}
$$
根据逆变换法, 
$$
F^{-1}(x) = -\frac{1}{\lambda} \ln(1-x).
$$
从而, \(Y = -\frac{1}{\lambda} \ln(1- U)\)服从参数为\(\lambda\)的指数分布.
又根据\(U \overset{d}{=} 1-U\), 从而\(Y = -\frac{1}{\lambda} \ln(U)\)也服从参数为\(\lambda\)的指数分布.
\end{proof}

% \medskip

% \printbibliography


\end{document}