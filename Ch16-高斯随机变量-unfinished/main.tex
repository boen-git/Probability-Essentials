\documentclass[UTF8, a4paper]{article}
\usepackage{ctex}
\usepackage{graphicx}
\usepackage[margin=2.5cm]{geometry}
\usepackage{subcaption}
\usepackage{amssymb}
\usepackage{amsthm}
\usepackage{amsmath}
\usepackage{enumerate}
\usepackage[backend=bibtex, style=alphabetic]{biblatex}
\usepackage{framed}
\usepackage{mathrsfs} 
\usepackage{xcolor}
\usepackage{arydshln}
\newcommand{\Perp}{\perp\!\!\!\!\perp}
\newtheorem{exercise}{Exercise \#16.}
\newtheorem*{proposition}{命题}
\newtheorem*{remark}{注}
\everymath{\displaystyle}

\addbibresource{my.bib}
\title{Chapter 16: 高斯随机变量}
\author{}
\date{Latest Update: \today}
\begin{document}
\maketitle

Gaussian积分:
$$
\int \exp\left\{-a(x+b)^2\right\} dx = \sqrt{\frac{\pi}{a}}.
$$


\begin{framed}
\begin{exercise}
设\(a \in \mathbb{R}^n, a \neq 0, \mu \in \mathbb{R}^n\).
设\(H\)是\(\mathbb{R}^n\)中的一个超平面,满足
$$
H=\left\{x \in \mathbf{R}^n:\langle x-\mu, a\rangle=0\right\} .
$$
证明: \(m_n(H) = 0\), 其中\(m_n\)是一个\(n\)维欧几里得空间\(\mathbb{R}^n\)上的Lebesgue测度.
并对任意\(\mathbb{R}^n\)上的Borel函数推导以下的关系:
$$
\int_H f(x) d x=\int_{-\infty}^{\infty} \ldots \int_{\infty}^{\infty} f\left(x_1, \ldots, x_n\right) 1_H\left(x_1, \ldots, x_n\right) d x_1 \ldots d x_n=0.
$$
\end{exercise}
\end{framed}

\begin{proof}
对第一部分, 由于\(H\)可以通过超平面\(x_n = 0\)的正交和平移变换得到.
即存在正交变换使得\(Ta = e_n = (0, ..., 1)^\top\), 从而令\(y = Tx\)有
$$
a^\top (x - \mu) = 0  \Rightarrow y_n = a^\top \mu.
$$
由于Lebesgue测度关于正交变换和平移变换是不变的, 从而只需证明\(m_n(H) = 0\)对\(H = \{x_n = 0\}\)成立.
由于\(\mathbb{Q}^{n} \sim \mathbb{Q}, \forall n \geq 1\), 整数. 
从而记\(\mathbb{Q}^{n-1}\times \{0\} = \{\mathbf{q}_1, \mathbf{q}_2, ...\}\), 
\(\forall \varepsilon >0\), 取以\(\mathbf{q}_k\)为中心, \(x_n\)分量长为\(\varepsilon\), 其余分量长为\(\frac{1}{2^{k/(n-1)}}\)的方体\(A_k\), 则根据有理数的稠密性, 
$$
H \subset \bigcup_{k=1}^{\infty} A_k.
$$
于是
$$
0 \leq m_n(H) \leq \sum_{k=1}^{\infty} m_n(A_k) = \sum_{k=1}^{\infty} \frac{\varepsilon}{2^{k}} = \varepsilon \to 0.
$$
从而\(m_n(H) = 0\).


对第二部分, 第一个等式因为积分的定义. 因此, 只需证当\(H\)是零测集时, 
$$
\int_H f(x) d x= 0.
$$
当\(f(x) = I_c(x)\)是简单函数时, 由于\(H\)是零测集, 有
$$
0 \leq \int_H f(x) d x=  \int_H I_c(x) d x = m_n(H \cap C) \leq m_n(H) = 0.
$$
从而根据线性性, 对于任意正\(f(x)\)都有\(\int_H f(x) d x\).
最后, 对一般的Borel函数\(f(x)\), 取\(f^+, f^-\), 分别成立等于0, 从而\(\int_H f(x) d x\).
\end{proof}


\begin{framed}
\begin{exercise}
记\(\mathcal{L}(Y) = \mathcal{N}(0, 1)\), 令\(a > 0\), 
$$
Z=\left\{\begin{array}{c}
Y \text { 若 }|Y| \leq a, \\
-Y \text { 若 }|Y|>a .
\end{array}\right.
$$
表明\(\mathcal{L}(Z) = N(0, 1)\).
\end{exercise}
\end{framed}

\begin{proof}
考察\(Z\)的特征函数, 
\begin{align*}
    \varphi_Z(u) &= \mathbb{E}\left\{e^{iuZ}\right\} \\
    &= \mathbb{E}\left\{e^{iuY}I_{\{|Y| \leq a\}}\right\} + \mathbb{E}\left\{e^{-iuY}I_{\{|Y| > a\}}\right\} \\
    &= \int_{|y| \leq a} e^{iuy} \frac{1}{\sqrt{2\pi}} e^{-y^2/2} dy + \int_{|y| > a} e^{-iuy} \frac{1}{\sqrt{2\pi}} e^{-y^2/2} dy \\
    &= \int_{|y| \leq a} e^{iuy} \frac{1}{\sqrt{2\pi}} e^{-y^2/2} dy + \int_{|-z| > a} e^{iuz} \frac{1}{\sqrt{2\pi}} e^{-z^2/2} dz \quad (\text{积分方向改变和Jacobian抵消}) \\
    &= \int_{\mathbb{R}} e^{iuy} \frac{1}{\sqrt{2\pi}} e^{-y^2/2} dy \\
    &= e^{-u^2/2}.
\end{align*}
由唯一性定理, \(\mathcal{L}(Z) = N(0, 1)\).

\end{proof}


\begin{framed}
\begin{exercise}
设\(X\)服从\(\mathcal{N}(0, 1)\), \(Z \Perp X\), 且满足\(P(Z = 1) = P(Z = -1) = \frac{1}{2}\).
令\(Y = XZ\). 证明\(Y \sim \mathcal{N}(0, 1)\). 但是\((X,Y)\)不是二元正态分布.
\end{exercise}
\end{framed}


\begin{proof}
考察\(Y\)的特征函数, 由于\(X\)与\(Z\)独立, \(\varphi_X(u) = e^{-\frac{1}{2}u^2}\).
\(\varphi_Z(u) = \frac{e^{iu} + e^{-iu}}{2} = cos(u)\).
从而用独立随机变量的特征函数
\begin{align*}
    \varphi_Y(u) &= \mathbb{E}\left\{e^{iuXZ}\right\} = \mathbb{E}\left\{e^{iuX}e^{iuZ}\right\} \\
    &= \frac{1}{2} \mathbb{E}\{e^{iuX}\} + \frac{1}{2} \mathbb{E}\{e^{-iuX}\} \\
    &= \frac{1}{2} \varphi_X(u) + \frac{1}{2} \varphi_X(-u) \\
    &= e^{-\frac{1}{2}u^2}.
\end{align*}
从而根据唯一性定理, \(Y \sim N(0, 1)\).

但是, 考察\(X\)与\(Y\)的联合特征函数,
\begin{align*}
    \varphi_{X,Y}(u,v) &= \mathbb{E}\left\{e^{i(uX + vY)}\right\} \\
    &= \mathbb{E}\left\{e^{i(uX + vXZ)}\right\} \\
    &= \mathbb{E}\left\{e^{iX(u + vZ)}\right\} \\
    &= \frac{1}{2} \mathbb{E}\left\{e^{iX(u + v)}\right\} + \frac{1}{2} \mathbb{E}\left\{e^{iX(u - v)}\right\} \\
    &= \frac{1}{2} \varphi_X(u + v) + \frac{1}{2} \varphi_X(u - v) \\
    &= e^{-\frac{1}{2}(u^2 + v^2)} \cos(uv).
\end{align*}
从而根据唯一性定理, \((X, Y)\)不是二元正态分布.
\end{proof}


\begin{framed}
\begin{exercise}
设\((X,Y)\)服从Gaussian分布, 均值\((\mu_X, \mu_Y)\), 协方差矩阵\(Q\)满足\(\det(Q) > 0\).
令\(\rho\)是相关系数
$$
\rho=\frac{\operatorname{Cov}(X, Y)}{\sqrt{\operatorname{Var}(X) \operatorname{Var}(Y)}} .
$$
证明: 当\(\rho \in (-1, 1)\), \((X,Y)\)的联合密度存在, 且为$$
\begin{array}{r}
f_{(X, Y)}(x, y)=\frac{1}{2 \pi \sigma_X \sigma_Y \sqrt{1-\rho^2}} \exp \left\{\frac { - 1 } { 2 ( 1 - \rho ^ { 2 } ) } \left(\left(\frac{x-\mu_X}{\sigma_X}\right)^2 -\frac{2 \rho\left(x-\mu_X\right)\left(y-\mu_Y\right)}{\sigma_X \sigma_Y}+\left(\frac{\left(y-\mu_Y\right)}{\sigma_Y}\right)^2\right)\right\}
\end{array}
$$
再证明当\(\rho = -1,1\)时, \((X,Y)\)的联合密度不存在.
\end{exercise}
\end{framed}


\begin{proof}
根据Cor 16.2, \((X,Y)\)的密度存在, 当且仅当\(Q\)是非退化的, 即\(\det(Q) \neq 0\).
$$
\det(Q) = \left| \begin{matrix}
    \sigma_X^2 & \rho \sigma_X \sigma_Y \\
    \rho \sigma_X \sigma_Y & \sigma_Y^2
\end{matrix}\right| = \sigma_X^2 \sigma_Y^2 (1 - \rho^2).
$$
因此, 当\(\rho \in (-1, 1)\), \((X,Y)\)的联合密度存在.
当\(\rho = \pm 1\), \(\det(Q) = 0\), \((X,Y)\)的联合密度不存在.
\end{proof}


\begin{framed}
\begin{exercise}
设\(\rho\)取值在\((-1, 1)\), 给定\(\mu_j, \sigma_j^2, j=1,2\).
构造\(X_1, X_2\)是正态的, 均值是\(\mu_1, \mu_2\), 方差是\(\sigma_1^2, \sigma_2^2\), 相关系数是\(\rho\).
\end{exercise}
\end{framed}
\begin{remark}
令\(Y_1, Y_2\)独立同标准正态分布. 令\(U_1 = Y_1,  U_2 = \rho Y_1 + \sqrt{1 -\rho^2} Y_2\).
接着令\(X_1 = \mu_1 + \sigma_1 U_1, X_2 = \mu_2 + \sigma_2 U_2\).
\end{remark}

\begin{proof}
一个直观的想法是, 直接令\(\mu = (\mu_1, \mu_2)^\top, \Sigma = \left(\begin{array}{cc}
    \sigma_1^2 & \rho \sigma_1 \sigma_2 \\
    \rho \sigma_1 \sigma_2 & \sigma_2^2
\end{array}\right)\), 则\((X_1, X_2)\)服从\(N(\mu, \Sigma)\).

提示中给出了一种构造方法, 令\(Y_1, Y_2\)独立同标准正态分布. 令\(U_1 = Y_1,  U_2 = \rho Y_1 + \sqrt{1 -\rho^2} Y_2\).
则 
$$
\binom{U_1}{U_2} \sim N\left(\binom{0}{0}, \left(\begin{array}{cc}
    1 & \rho \\
    \rho & 1
\end{array}\right)\right).
$$
令\(X_1 = \mu_1 + \sigma_1 U_1, X_2 = \mu_2 + \sigma_2 U_2\), 则
$$
\binom{X_1}{X_2} \sim N\left(\binom{\mu_1}{\mu_2}, \left(\begin{array}{cc}
    \sigma_1^2 & \rho \sigma_1 \sigma_2 \\
    \rho \sigma_1 \sigma_2 & \sigma_2^2
\end{array}\right)\right).
$$
这样就构造了满足条件的\(X\).
\end{proof}



\begin{framed}
\begin{exercise}
假设\(X\)服从\(\mathbb{R}^n\)上的Gaussian分布\(N(\mu, \Sigma)\), 其中\(\det(\Sigma) > 0\).
证明: 存在矩阵\(B\)使得\(Y = B(X - \mu)\)服从\(N(0, I)\), 其中\(I\)是\(n\)阶单位矩阵.
\end{exercise}
\end{framed}

\begin{remark}
这个题目表明: 任何非退化的Gaussian随机变量都可以通过线性变换得到标准正态分布.
\end{remark}

\begin{proof}
由于\(\Sigma\)是正定的, 从而根据特征分解, 存在正交矩阵\(A^\top = A^{-1}\)使得\(\Sigma = A \Lambda A^\top\).
其中\(\Lambda = \operatorname{diag}\{\lambda_1, ..., \lambda_n\}, \lambda_i > 0, \forall i\). 

令\(B = A^\top \Lambda^{1/2}\), 其中\(\Lambda^{1/2} = \operatorname{diag}\{\lambda_1^{1/2}, ..., \lambda_n^{1/2}\}\), 于是 
$$
\mathbb{E}\{Y\} = \mathbb{E}\{B(X - \mu)\} = B(\mathbb{E}\{X\} - \mu) = 0_{n}. 
$$
以及
$$
\operatorname{Var}(Y) = B \Sigma B^\top = I_{n\times n}.
$$
由定理16.1, \(Y\)服从\(N(0, I)\).
\end{proof}


\begin{framed}
\begin{exercise}
设\(X\)服从Gaussian分布, 令
$$
Y=\sum_{j=1}^n a_j X_j
$$
其中\(X = (X_1, ..., X_n)^\top\)服从\(N(\mu, \Sigma)\). 证明:
\(Y\)服从一元Gaussian分布\(N(\mu, \sigma^2)\), 其中
$$
\mu=\sum_{j=1}^n a_j E\left\{X_j\right\}
$$
以及
$$
\sigma^2=\sum_{j=1}^n a_j^2 \operatorname{Var}\left(X_j\right)+2 \sum_{j<k} a_j a_k \operatorname{Cov}\left(X_j, X_k\right).
$$
\end{exercise}
\end{framed}

\begin{proof}
由于\(X\)服从\(N(\mu, \Sigma)\), \(Y\)是一个线性组合, 从而根据定义, \(Y\)也服从正态分布.
又因为\(\mathbb{E}(Y) = \mu, \operatorname{Var}(Y) = \sigma^2\), 根据特征函数, \(Y\)服从一元Gaussian分布\(N(\mu, \sigma^2)\).

$$
\begin{aligned}
\mathbb{E}\{Y\} & =\mathbb{E}\left\{\sum_{j=1}^n a_j X_j\right\}=\sum_{j=1}^n a_j \mathbb{E}\left\{X_j\right\}, \\
\operatorname{Var}(Y) & =\operatorname{Cov}\left(\sum_{j=1}^n a_j X_j, \sum_{k=1}^n a_k X_k\right) \\
& =\sum_{j=1}^n \sum_{k=1}^n a_j a_k \operatorname{Cov}\left(X_j, X_k\right) \\
& =\sum_{j=1}^n  \operatorname{Var}\left(X_j\right)+\sum_{j \neq k} a_j a_k \operatorname{Cov}\left(X_j, X_k\right) \\
& =\sum_{j=1}^n  \operatorname{Var}\left(X_j\right)+2 \sum_{j<k} a_j a_k \operatorname{Cov}\left(X_j, X_k\right)
\end{aligned}
$$
\end{proof}

\begin{remark}
综合16.6和16.7, 我们给出线性变换定理: 设\(X \sim N_p(\mu, \Sigma)\), \(B\)为\(s \times p\)常数矩阵, \(d\)为\(s\)维常数向量, 则\(Y = BX + d\)服从\(N_s(B\mu + d, B\Sigma B^\top)\).

由于\(Y\)的特征函数 
$$
\begin{aligned}
\varphi_Y(u) &= \mathbb{E}\{e^{i u^\top Y}\} \\
&= e^{i u^\top d} \mathbb{E}\{e^{i u^\top BX}\} \\
&= e^{i u^\top d} \varphi_X(B^\top u) \\
&= \exp\left\{i \langle u, d \rangle + i \langle u, BX \rangle - \frac{1}{2} \langle u, B \Sigma B^\top u \rangle\right\}.
\end{aligned}
$$
\end{remark}


\begin{framed}
\begin{exercise}
设\((X,Y)\)服从二元正态分布\(N(\mu, \Sigma)\), 其中
$$
\Sigma=\left(\begin{array}{cc}
\sigma_X^2 & \rho \sigma_X \sigma_Y \\
\rho \sigma_X \sigma_Y & \sigma_Y^2
\end{array}\right)
$$
\(\rho\)是相关系数, 满足\(|\rho| < 1\).(即\(\det(\Sigma) > 0\).)
则\((X,Y)\)的联合密度为\(f\), 证明它们的条件密度\(f_{Y|X}(y)\)是一元正态分布\(N\left(\mu_Y + \rho \frac{\sigma_Y}{\sigma_X}(x - \mu_X), \sigma_Y^2(1 - \rho^2)\right)\).
\end{exercise}
\end{framed}

\begin{proof}
    已经在12.3中证明过. 证明通过展开联合分布, 求出\(X\)的边际密度, 最后通过条件分布的定义求出条件密度.
\end{proof}



\begin{framed}
\begin{exercise}
设\(X\)服从\(\mathcal{N}(\mu, \Sigma)\), 其中\(\mu = (1, 1)^\top, \Sigma=\left(\begin{array}{cc}
    3 & 1 \\
    1 & 2
    \end{array}\right)\).求出\(Y|Z\)的分布, 其中\(Y = X_1 + X_2, Z = X_1 - X_2\).
\end{exercise}
\end{framed}


\begin{proof}
由于 
$$
\binom{Y}{Z} = \binom{X_1 + X_2}{X_1 - X_2} = \left(\begin{array}{cc}
    1 & 1 \\
    1 & -1  
\end{array}\right) \binom{X_1}{X_2},
$$
从而 
$$
\binom{Y}{Z} \sim N\left(\left(\begin{array}{cc}
    1 & 1 \\
    1 & -1  
\end{array}\right)\binom{1}{1}, \left(\begin{array}{cc}
    1 & 1 \\
    1 & -1  
\end{array}\right)
\left(\begin{array}{cc}
    3 & 1 \\
    1 & 2
    \end{array}\right)
\left(\begin{array}{cc}
    1 & 1 \\
    1 & -1  
\end{array}\right)
\right).
$$
计算得
$$
\binom{Y}{Z} \sim N\left(\binom{2}{0}, \left(\begin{array}{cc}
    7 & 1 \\ 
    1 & 3
\end{array}\right)\right).
$$
由于条件分布为 
$$
Y|Z \sim N(\mu_{1\cdot 2}, \Sigma_{11\cdot 2}), 
$$
其中, 
$$
\mu_{1\cdot 2} = \mu_1 + \Sigma_{12}\Sigma_{22}^{-1}(z - \mu_2) = 2 + \frac{1}{3}(z - 0) = \frac{z}{3} + 2,
$$
以及
$$
\Sigma_{11\cdot 2} = \Sigma_{11} - \Sigma_{12}\Sigma_{22}^{-1}\Sigma_{21} = 7 - \frac{1}{3} = \frac{20}{3}.
$$
从而, 
$$
f_{Z=0}(y)=\frac{1}{\sqrt{2 \pi} \sqrt{\frac{20}{3}}} \exp \left\{-\frac{1}{2} \frac{(y-2)^2}{\frac{20}{3}}\right\}
$$
\end{proof}


\begin{framed}
\begin{exercise}
设\(X\)服从\(\mathcal{N}(\mu, \Sigma)\), 满足\(\det(\Sigma) > 0\).
证明: 任意维元素的条件分布还是多元正态分布.
\end{exercise}
\end{framed}

\begin{remark}
条件期望和条件方差的另一种表述是用Schur补.
\end{remark}

\begin{proof}
设\(X = \binom{X^{(1)}}{X_2}\), 其中, \(X \in \mathbb{R}^p, X^{(1)} \in \mathbb{R}^{(r)}, X^{(2)} \in \mathbb{R}^{p-r}\).
不妨考虑\(X^{(1)}\)是我们要求的条件变量, \(X^{(2)}\)是我们要求的条件变量. 由于Gaussian分布的前提是半正定的, \(\det(\Sigma) > 0\), 从而\(\Sigma\)正定,  从而\(\Sigma_{22}\)正定, 从而\(\Sigma_{22}^{-1}\)存在. 从而考察非奇异线性变换. 令
$$
Z = \binom{Z^{(1)}}{Z^{(2)}} = \binom{X^{(1)} - \Sigma_{12} \Sigma_{22}^{-1} X^{(2)}}{X^{(2)}} = 
\left(\begin{array}{c;{2pt/2pt}c}
    I_r & -\Sigma_{12}\Sigma_{22}^{-1} \\ \hdashline[2pt/2pt]
    0 & I_{p-r}
\end{array}\right) \binom{X^{(1)}}{X^{(2)}} \triangleq B \binom{X^{(1)}}{X^{(2)}}.
$$
根据线性变换法则, 
$$
Z \sim N_p\left(\left(\begin{array}{c}
    \mu_1 - \Sigma_{12}\Sigma_{22}^{-1}\mu_2 \\
    \mu_2
\end{array}\right), \left(\begin{array}{c;{2pt/2pt}c}
    \Sigma_{11} - \Sigma_{12}\Sigma_{22}^{-1}\Sigma_{21} & 0 \\ \hdashline[2pt/2pt]
    0 & \Sigma_{22}
\end{array}\right)\right).
$$
从而\(Z^{(1)}\)与\(Z^{(2)}\)相互独立. \(Z\)的联合密度为 
$$
g(z^{(1)}, z^{(2)}) = g_1(z^{(1)}) g_2(z^{(2)}) = g_1(z^{(1)}) f_2(z^{(2)})
$$
其中, 由于\(Z^{(2)} = X^{(2)}\), \(f_2(z^{(2)})\)是\(X^{(2)}\)的密度函数. 

由于\(Z = BX\), 使用积分变换公式: 
$$
\begin{aligned}
    f(x^{(1)}, x^{(2)}) =& g(BX) \cdot J(z \to x) \\
    =& g_1(x^{(1)} - \Sigma_{12} \Sigma_{22}^{-1} x^{(2)}) g_2(x^{(2)}) \cdot |\det(B^\top)|_+ \\
    =& g_1(x^{(1)} - \Sigma_{12} \Sigma_{22}^{-1} x^{(2)}) f_2(x^{(2)})  \\
\end{aligned}
$$
由于 
$$
Z^{(1)} \sim N_r(\mu^{(1)} - \Sigma_{12} \Sigma_{22}^{-1} \mu^{(2)}, \Sigma_{11} - \Sigma_{12}\Sigma_{22}^{-1}\Sigma_{21}),
$$
从而可以写出\(f(x^{(1)}|x^{(2)})\).  
$$
\begin{aligned}
f_1\left(x^{(1)} \mid\right. & \left.x^{(2)}\right)=\frac{f\left(x^{(1)}, x^{(2)}\right)}{f_2\left(x^{(2)}\right)}=g_1\left(x^{(1)}-\Sigma_{12} \Sigma_{22}^{-1} x^{(2)}\right) \\
= & \frac{1}{(2 \pi)^{r / 2}\left|\Sigma_{11 \cdot 2}\right|^{1 / 2}} \exp \left[-\frac{1}{2}\left(x^{(1)}-\Sigma_{12} \Sigma_{22}^{-1} x^{(2)}\right.\right. \\
& \left.-\left(\mu^{(1)}-\Sigma_{12} \Sigma_{22}^{-1} \mu^{(2)}\right)\right)^{\prime} \Sigma_{11 \cdot 2}^{-1}\left(x^{(1)}-\Sigma_{12} \Sigma_{22}^{-1} x^{(2)}\right. \\
& \left.\left.-\left(\mu^{(1)}-\Sigma_{12} \Sigma_{22}^{-1} \mu^{(2)}\right)\right)\right] \\
= & \frac{1}{(2 \pi)^{r / 2}\left|\Sigma_{11 \cdot 2}\right|^{1 / 2}} \exp \left[-\frac{1}{2}\left(x^{(1)}-\mu_{1 \cdot 2}\right)^{\prime} \Sigma_{11 \cdot 2}^{-1}\left(x^{(1)}-\mu_{1 \cdot 2}\right)\right]
\end{aligned}
$$


其中

$$
\begin{gathered}
\mu_{1 \cdot 2}=\mu^{(1)}+\Sigma_{12} \Sigma_{22}^{-1}\left(x^{(2)}-\mu^{(2)}\right), \\
\Sigma_{11 \cdot 2}=\Sigma_{11}-\Sigma_{12} \Sigma_{22}^{-1} \Sigma_{21} .
\end{gathered}
$$
\end{proof}




\begin{framed}
\begin{exercise}
设\((X,Y)\)有联合密度:
$$
f_{(X, Y)}(x, y)=c \exp \left\{-\left(1+x^2\right)\left(1+y^2\right)\right\}, \quad-\infty<x, y<\infty,
$$
其中, 选择特定的\(c\)使得\(f_{(X, Y)}(x, y)\)是一个联合密度.
证明: \(f\){\it 不是}二元正态的密度函数, 但是\(X\)和\(Y\)的边际密度都是正态密度.
\end{exercise}
\end{framed}

\begin{remark}
这表明习题16.10的逆命题不成立.
\end{remark}


\begin{proof}
这里\(c\)的计算不是简单的, 这里直接取\(c^{-1} = \iint \exp \left\{-\left(1+x^2\right)\left(1+y^2\right)\right\} \, dx dy\).
考察\(X\)的边际密度,
\begin{align*}
f_X(x) &= \int_{-\infty}^\infty f_{(X,Y)}(x, y) \, dy \\
&= c \int_{-\infty}^\infty \exp \left\{-\left(1+x^2+y^2 + x^2 y^2\right)\right\} \, dy \\
&= c \exp\left( - 1 - x^2\right)\int_{-\infty}^\infty \exp \left\{-(x^2 + 1) y^2\right\} \, dy \\
&= c \exp\left( - 1 - x^2\right) \sqrt{\frac{\pi}{x^2 + 1}}.
\end{align*}

从而条件分布是
\begin{align*}
    f_{Y|X}(y|x) &= \frac{f_{(X,Y)}(x, y)}{f_X(x)} \\
    &= \frac{\sqrt{x^2 + 1}}{\sqrt{\pi}} \exp\left\{-\left(y^2 - x^2 y^2\right)\right\} \\
    &= \left\{2\pi \cdot \frac{1}{2(x^2 + 1)}\right\}^{-\frac{1}{2}} \exp \left\{-\frac{y^2}{2 \cdot \frac{1}{2(x^2 + 1)}}\right\}.
\end{align*}
从而\(Y|X = x\)是正态分布\(\mathcal{N}\left(0, \frac{1}{2(x^2 + 1)}\right)\). 由于对称性, 
\(X|Y = y\)服从\(\mathcal{N}\left(0, \frac{1}{2(y^2 + 1)}\right)\).

从而\(X\)和\(Y\)的条件密度都是正态密度. 但是\(f_{(X,Y)}(x, y)\)不是二元正态的密度函数. 因为二元正态密度不会出现\(x^2 y^2\)这种项.
\end{proof}


\begin{framed}
\begin{exercise}
设\((X,Y)\)是二元正态分布, 有相关系数\(\rho\), 均值为\((0,0)^\top\). 证明:
当\(|\rho| <1\), 则\(Z = \frac{X}{Y}\)是Cauchy随机变量, 参数\(\alpha = \rho\frac{\sigma_X}{\sigma_Y}, \beta = \frac{\sigma_X}{\sigma_Y}\sqrt{1 - \rho^2}\).
我们总结: 中心化的二元正态分布的比是Cauchy分布.
\end{exercise}
\end{framed}

\begin{remark}
    这个结果在例12.5中当\(X,Y\)是独立的情况已经证明.
\end{remark}


\begin{framed}
设\(X, Y\)独立同\(\mathcal{N}(0, \sigma^2)\). 令\(Z = \sqrt{X^2+ Y^2}, W = \begin{cases}
    \frac{X}{Y}, & Y \neq 0, \\
    0, & Y = 0.
\end{cases}\). 现在我们要求\(Z, W\)的联合密度.

考察\(g:(x,y) \mapsto (z,w), \mathbb{R}^2 \to \mathbb{R}_+ \times \mathbb{R}\)的映射:
$$
g(x,y) = \left(\sqrt{x^2 + y^2}, \frac{x}{y}\right) = (z, w).
$$
显然\(g\)本身不是单射. 因此我们考虑\(g\)的单射区域.  
记\(S_0 = \{(x,y): x = 0, y\in \mathbb{R}\} \cup \{(x,y): y = 0, x\in \mathbb{R}\}\).
\(S_1 = \{(x, y): y > 0\}\)为第一, 二象限, \(S_2 = \{(x, y): y < 0\}\)为第三, 四象限.
于是\(\mathbb{R}^2 = \sum_{i=0}^{2}S_i\), 且\(m_2(S_0) = 0\), \(S_1, S_2\)是两片单射区域.

在\(S_1\)上, \(g\)是单射, 且$$
g^{-1}(z,w) = \left(\frac{zw}{\sqrt{1+w^2}}, \frac{z}{\sqrt{1+w^2}}\right).
$$
此时, Jacobian:
$$
J_1 = \left|\begin{matrix}
    \frac{w}{\sqrt{1+w^2}} & \frac{1}{\sqrt{1+w^2}} \\
    \frac{z}{{(1+w^2)^{\frac{3}{2}}}} & -\frac{wz}{(1+w^2)^{\frac{3}{2}}}
\end{matrix}\right| = \frac{-z}{1+w^2}.
$$
在\(S_2\)上, \(g\)是单射, 且$$
g^{-1}(z,w) = \left(-\frac{zw}{\sqrt{1+w^2}}, -\frac{z}{\sqrt{1+w^2}}\right).
$$
此时, Jacobian:
$$
J_2 = \left|\begin{matrix}
    -\frac{w}{\sqrt{1+w^2}} & -\frac{1}{\sqrt{1+w^2}} \\
    -\frac{z}{{(1+w^2)^{\frac{3}{2}}}} & \frac{wz}{(1+w^2)^{\frac{3}{2}}}
\end{matrix}\right| = \frac{-z}{1+w^2}.
$$
于是, \(Z, W\)的联合密度为
$$
f_{(Z,W)}(z,w) = \frac{z}{1+w^2}\left\{f_{(X,Y)}\left(\frac{zw}{\sqrt{1+w^2}}, \frac{z}{\sqrt{1+w^2}}\right) + f_{(X,Y)}\left(\frac{-zw}{\sqrt{1+w^2}}, \frac{-z}{\sqrt{1+w^2}}\right)\right\}.
$$
从而, 
$$
f_{(Z,W)}(z,w) = \frac{2z}{1+w^2} (2\pi\sigma^2)^{-1} \exp\left\{-\frac{z^2}{2\sigma^2}\right\} \mathbb{I}_{\{Z >0\}} = \frac{1}{\pi \sigma^2}\left(\frac{1}{1+w^2}\right)\left(z e^{-\frac{z^2}{2\sigma^2}} \mathbb{I}_{\{Z > 0\}}\right).
$$
从而可以推断出\(Z,W\)独立, 且\(W\)服从参数为\(0, 1\)的Cauchy分布.
$$
f_W(w) = \frac{1}{\pi(1+w^2)}, \quad w \in \mathbb{R}.
$$
以及\(Z\)服从参数为\(\sigma^2\)的Rayleigh分布. 
$$
f_Z(z) = \frac{z}{\sigma^2} e^{-\frac{z^2}{2\sigma^2}} \mathbb{I}_{\{z > 0\}}, \quad z \in \mathbb{R}_+.
$$
\end{framed}

\begin{proof}
使用上面的变换方法. 此时, \((X,Y)\)的联合密度为
$$
f_{(X, Y)}(x, y)=\frac{1}{2 \pi \sigma_X \sigma_Y \sqrt{1-\rho^2}} \exp \left\{\frac { - 1 } { 2 ( 1 - \rho ^ { 2 } ) } \left(\left(\frac{x}{\sigma_X}\right)^2 -\frac{2 \rho xy}{\sigma_X \sigma_Y}+\left(\frac{y}{\sigma_Y}\right)^2\right)\right\}
$$
于是, \((Z,W)\)的联合密度为 
\begin{align*}
f_{(Z,W)}(z,w) &= \frac{z}{1+w^2}\left\{f_{(X,Y)}\left(\frac{zw}{\sqrt{1+w^2}}, \frac{z}{\sqrt{1+w^2}}\right) + f_{(X,Y)}\left(\frac{-zw}{\sqrt{1+w^2}}, \frac{-z}{\sqrt{1+w^2}}\right)\right\} \\
&= \frac{2z}{1+w^2}\left\{f_{(X,Y)}\left(\frac{zw}{\sqrt{1+w^2}}, \frac{z}{\sqrt{1+w^2}}\right)  \right\} \\
&= \frac{2z}{1+w^2} \left[\frac{1}{2 \pi \sigma_X \sigma_Y \sqrt{1-\rho^2}} \exp \left\{\frac { - 1 } { 2 ( 1 - \rho ^ { 2 } ) } \left(\left(\frac{zw}{\sigma_X\sqrt{1+w^2}}\right)^2 -\frac{2 \rho \frac{zw}{\sqrt{1+w^2}} \frac{z}{\sqrt{1+w^2}}}{\sigma_X \sigma_Y}+\left(\frac{z}{\sigma_Y\sqrt{1+w^2}}\right)^2\right)\right\}\right] \\
&= \frac{2z}{1+w^2} \left[\frac{1}{2 \pi \sigma_X \sigma_Y \sqrt{1-\rho^2}} \exp \left\{\frac{-1}{2(1-\rho^2)} \left[\frac{z^2 w^2}{ \sigma_X^2 (1+w^2)} - \frac{2\rho z^2 w}{\sigma_X \sigma_Y (1 + w^2)} + \frac{z^2}{\sigma_Y^2(1+w^2)}\right]\right\}\right] \\
&= \frac{2z}{1+w^2} \left[\frac{1}{2 \pi \sigma_X \sigma_Y \sqrt{1-\rho^2}} \exp \left\{\frac{-1}{2(1-\rho^2)\sigma_X^2 \sigma_Y^2 (1+w^2)} \left[{z^2 w^2\sigma_Y^2} - {2\rho z^2 w \sigma_X \sigma_Y} + {z^2 \sigma_X^2}\right]\right\}\right] \\
\end{align*}
关于\(z\)求积分. 令
$$
p\triangleq \pi \sigma_X\sigma_Y \sqrt{1-\rho^2}(1+w^2), \quad q \triangleq -\frac{w^2 \sigma_Y^2 - 2\rho w \sigma_X \sigma_Y + \sigma_X^2}{2\sigma_X^2 \sigma_Y^2 (1 - \rho^2)(1 + w^2)}
$$
于是 
$$
f_W(w) = \int_{0}^{\infty} \frac{z}{p} \exp(qz^2) \, dz = -\frac{1}{2pq}.
$$
即 \begin{align*}
    f_W(w) &= \frac{1}{\pi \cdot \frac{w^2 \sigma_Y^2 - 2\rho w \sigma_X \sigma_Y + \sigma_X^2}{\sigma_X \sigma_Y \sqrt{1-\rho^2}}} \\
    &= \frac{1}{{\pi }} \cdot \frac{\sigma_X \sqrt{1-\rho^2}}{\sigma_Y} \cdot \frac{1}{\frac{\sigma_X^2(1 - \rho^2)}{\sigma_Y^2} + \left(w^2 - 2\rho\frac{\sigma_X}{\sigma_Y} w + \rho^2 \frac{\sigma_X^2}{\sigma_Y^2}\right)} \\
    & = \frac{1}{\beta \pi} \cdot \frac{1}{1+ \left(\frac{x-\alpha}{\beta}\right)^2}
\end{align*}
其中, \(\alpha = \rho\frac{\sigma_X}{\sigma_Y}, \beta = \frac{\sigma_X}{\sigma_Y}\sqrt{1 - \rho^2}\).
\end{proof}




\begin{framed}
\begin{exercise}
令\((X,Y)\)是二元正态变量, 均值是\(0\), 相关系数\(\rho\).
设\(\beta\)满足
$$
\cos \beta=\rho \quad(0 \leq \beta \leq \pi)
$$
再证明:
$$
P\{X Y<0\}=\frac{\beta}{\pi} .
$$
\end{exercise}
\end{framed}

\begin{remark}
在习题16.12中, 我们已经证明了若\(Z = \frac{X}{Y}, z = \rho\frac{\sigma_X}{\sigma_Y}\),
则 
$$
F_Z(z)=\frac{1}{2}+\frac{1}{\pi} \operatorname{Arctan}\left(\frac{z \sigma_Y-\rho \sigma_X}{\sigma_X \sqrt{1-\rho^2}}\right) .
$$
令\(\alpha = \arcsin(\rho)\), 其中\(\left(-\frac{\pi}{2} \leq \alpha \leq \frac{\pi}{2}\right)\), 
并且用\(\arctan\frac{\rho}{\sqrt{1 - \rho^2}} = \arcsin \rho\), 证明\(P(XY < 0) = \frac{1}{2} - \frac{\alpha}{\pi}\).
\end{remark}

\begin{proof}
若Cauchy分布的密度函数为 
$$
f_Z(z|\alpha, \beta) = \frac{1}{\beta \pi} \cdot \frac{1}{1 + \left(\frac{z - \alpha}{\beta}\right)^2}, 
$$
则Cauchy分布的分布函数为
$$
F_Z(z) = \frac{1}{2} + \frac{1}{\pi} \arctan \left(\frac{z - \alpha}{\beta}\right), \quad z\in \mathbb{R}.
$$
从而,
$$
P(XY < 0) = P\left(\frac{X}{Y} < 0\right) = F_Z(0) = \frac{1}{2} - \frac{1}{\pi}\arctan\left(\frac{\alpha}{\beta}\right).
$$
而由上一题, 
$$
\frac{\alpha}{\beta} = \frac{\rho}{\sqrt{1 - \rho^2}}.
$$
从而
$$
P(XY < 0 ) = \frac{1}{2} - \frac{\arcsin(\rho)}{\pi} = \frac{1}{\pi} \left\{\frac{\pi}{2} - \arcsin(\rho)\right\} = \frac{\arccos(\rho)}{\pi}.
$$
但是由于\(\rho \in (-1, 1)\), 于是\(\arcsin(\rho) \in (-\pi/2, \pi/2)\).
从而
$$
P(XY < 0) = \frac{\beta}{\pi}.
$$
\end{proof}




\begin{framed}
\begin{exercise}
设\((X,Y)\)满足14.13中的条件, 证明:
$$
\begin{aligned}
& P\{X>0, Y>0\}=P\{X<0, Y<0\}=\frac{1}{4}+\frac{\alpha}{2 \pi} \\
& P\{X>0, Y<0\}=P\{X<0, Y>0\}=\frac{1}{4}-\frac{\alpha}{2 \pi}
\end{aligned}
$$
\end{exercise}
\end{framed}

\begin{proof}
由于\(1 = P(XY > 0) + P(XY < 0) + P(XY = 0) = P(XY < 0) + P(XY > 0)\). 以及
$$
P(XY < 0) = P(X < 0, Y > 0) + P(X > 0,  Y < 0) = \frac{1}{2} - \frac{\arcsin(\rho)}{\pi}, 
$$
从而只需证, 
$$
P(X < 0, Y > 0) = P(X > 0, Y < 0), \quad P(X > 0, Y > 0) = P(X < 0, Y < 0).
$$
而这, 由于\((X,Y)\)的联合密度函数满足
$$
f(x, y) = f(-x, -y)
$$
于是 
$$
\int_{0}^{\infty} \int_{0}^{\infty} f(x, y) dx dy = \int_{0}^{\infty} \int_{0}^{\infty} f(-x, -y) dx dy = \int_{-\infty}^{0} \int_{-\infty}^{0} f(x, y) dx dy.
$$
于是本题得证.




\end{proof}



\begin{framed}
\begin{exercise}
令\((X,Y)\)服从二元正态分布, 密度函数是
$$
f_{(X, Y)}(x, y)=\frac{1}{2 \pi \sigma_X \sigma_Y \sqrt{1-\rho^2}} e^{-\frac{1}{2\left(1-\rho^2\right)}\left(\frac{x^2}{\sigma_X^2}-\frac{2 \rho x y}{\sigma_X \sigma_Y}+\frac{y^2}{\sigma_Y^2}\right)} .
$$
证明:
\begin{enumerate}[a)]
    \item \(\mathbb{E}\{XY\} = \rho \sigma_X \sigma_Y\).
    \item \(\mathbb{E}\{X^2Y^2\} = \mathbb{E}\{X^2\}\mathbb{E}\{Y^2\} - 2 (\mathbb{E}\{XY\})^2\).
    \item \(\mathbb{E}\{|XY|\} = \frac{2\sigma_X \sigma_Y}{\pi}(\cos\alpha + \alpha \sin\alpha)\), 其中\(\alpha\)由\(\sin\alpha = \rho\left(-\frac{\pi}{2} \leq \alpha \leq \frac{\pi}{2}\right)\)给出.
\end{enumerate}
\end{exercise}
\end{framed}


\begin{proof}
\begin{enumerate}[a)]
    \item 由定义, \(\rho \sigma_X \sigma_Y = \operatorname{cov}(X,Y) = \mathbb{E}\{XY\} - \mathbb{E}\{X\} \mathbb{E}\{Y\} = \mathbb{E}\{XY\}\).
    \item 证明需要用到特征函数.
    \item 
\end{enumerate}
\end{proof}



\begin{framed}
\begin{exercise}
设\((X,Y)\)是二元正态分布, 相关系数是\(\rho\), \(\sigma_X^2 = \sigma_Y^2\).
证明: \(X\)和\(Y - \rho X\)独立.
\end{exercise}
\end{framed}

\begin{proof}
只需证: \(\operatorname{cov}(X, Y - \rho X) = 0\).
$$
\operatorname{cov}(X, Y - \rho X) = \operatorname{cov}(X, Y) - \rho \operatorname{cov}(X, X) = \rho \sigma_X \sigma_Y - \rho \sigma_X^2= 0.
$$
\end{proof}


\begin{framed}
\begin{exercise}
设\(X\)服从\(\mathcal{N}(\mu, \Sigma)\), 满足\(\det(\Sigma) > 0\).
\(X\)是取值在\(\mathbb{R}^n\)中的.
证明:
$$
(X-\mu)^\top \Sigma^{-1}(X-\mu)\quad  \sim \quad \chi^2(n) .
$$
\end{exercise}
\end{framed}

\begin{proof}
由于\(\Sigma\)正定, 则存在正定的平方根矩阵\(\Sigma^{\frac{1}{2}}\), 使得\(\Sigma = \Sigma^{\frac{1}{2}}\Sigma^{\frac{1}{2}}\).
考察 
$$
(X-\mu)^\top \Sigma^{-1/2} \Sigma^{-1/2}(X-\mu) = \left\{\Sigma^{-1/2}(X-\mu)\right\}^\top \left\{\Sigma^{-1/2}(X-\mu)\right\} ,
$$
而根据线性变换定理, 
$$
Z\triangleq \Sigma^{-1/2}(X-\mu) \sim \mathcal{N}(0, I_n).
$$
从而\(Z_i\)独立同\(\mathcal{N}(0, 1)\), 于是\(Z_i^2\)独立同\(\chi^2(1)\).
根据\(\chi^2\)分布的可加性(特征函数), 
$$
\sum_{i=1}^{n}Z_i^2 = \sum_{i=1}^{n}\left(\Sigma^{-1/2}(X-\mu)\right)_i^2 = (X-\mu)^\top \Sigma^{-1}(X-\mu) \sim \chi^2(n).
$$
\end{proof}


后面三道题是线性模型的一些应用, 与主干无关, 这里暂时跳过.
\begin{exercise}
设\(X_1, ..., X_n\)独立同\(\mathcal{N}(0, \sigma^2)\), 并且令
$$
\bar{x}=\frac{1}{n} \sum_{j=1}^n X_j \text { 和 } S^2=\frac{1}{n-1} \sum_{j=1}^n\left(X_j-\bar{x}\right)^2 .
$$
在习题15.13中, 已经证明\(\bar{x}\)和\(S^2\)是独立的.
证明: $$
\sum_{j=1}^n X_j^2=\sum_{j=1}^n\left(X_j-\bar{x}\right)^2+n \bar{x}^2
$$
并推导出\(\frac{(n-1)S^2}{\sigma^2}\sim \chi_{n-1}^2\), \(\frac{n \bar{x}^2}{\sigma^2} \sim \chi_1^2\).
\end{exercise}


\begin{exercise}
设\(\varepsilon_1,...,\varepsilon_n\)独立同\(\mathcal{N}(0, \sigma^2)\), 假设\(Y_i =  \alpha + \beta x_i + \varepsilon_i, 1 \leq i \leq n\).
假设\(x_i\)不全相等, 令\(\bar{x} = \frac{1}{n}\sum_{i = 1}^n x_i\).
定义回归残差是:
$$
\hat{\varepsilon}_i=Y_i-A-B x_i.
$$
\begin{enumerate}[a)]
    \item 证明: \(\mathbb{E}\{\hat{\varepsilon}_i\} = 0, 1\leq i\leq n\).
    \item 证明: $$
\operatorname{Var}\left(\hat{\varepsilon}_i\right)=\sigma^2\left(\frac{n-1}{n}-\frac{\left(x_i-\bar{x}\right)^2}{\sum_{i=1}^n\left(x_i-\bar{x}\right)^2}\right)
$$
\end{enumerate}
\end{exercise}

\begin{exercise}
考虑习题16.19定义的误差和残差. 假设\(\sigma\)是已知的, 定义
$$
\hat{\sigma}^2=\frac{1}{n} \sum_{i=1}^n \hat{\varepsilon}_i^2
$$
证明: \(\mathbb{E}\{\hat{\sigma}^2\} = \frac{n-2}{n}\sigma^2\).
\end{exercise}
\begin{remark}
由于\(\mathbb{E}\{\hat{\sigma}^2\} \neq \sigma^2\), 称\(\hat{\sigma}^2\)是\(\sigma^2\)的有偏估计.
于是\(\sigma^2\)的无偏估计是\(\frac{n}{n-2}\hat{\sigma}^2\).
\end{remark}


\begin{exercise}
在习题16.19, 16.20的条件下, 证明:\((A, B), S^2\)是独立的.
\end{exercise}


\begin{framed}
线性模型.
$$
Y_i = \alpha + \beta x_i + \varepsilon_i, 1 \leq i \leq n.
$$
其中\(\alpha, \beta, x_i\)是常数.

一个典型的模型是考察个体在测量\(\alpha + \beta x_i\), 但是在测量时有测量误差\(\varepsilon_i\).
根据中心极限定理, 可以假设\(\{\varepsilon_i\}\)服从多元正态分布. 可以称\(x_i\)为预测变量, \(Y_i\)为响应变量.

假设\(\mathbb{E}\{\varepsilon_i\} = 0, 1\leq i\leq n\), 对等式两边取期望, 有 
$$
\mathbb{E}\{Y_i\} = \alpha + \beta x_i.
$$
一般来说, 我们想要知道\(Y_i\)与\(x_i\)之间的线性关系, 如果\(\varepsilon\)不存在当然知道这样的关系. 换句话说, 目标是在知道\(x_i, Y_i\)下, 找出\(\alpha, \beta\).
首先, 我们要假设\(x_i\)是不全相等的, 否则我们最好给出一个常数估计\(\alpha + \beta x_1\), 但是此时无法识别出\(\alpha, \beta\). 
\(x_i\)不全相等, 当且仅当\(\sum_{i=1}^{n}(x_i - \bar{x})^2 > 0\), 其中\(\bar{x} = \frac{1}{n}\sum_{i=1}^{n}x_i\).

假设\(\varepsilon_i\)独立同分布于\(\mathcal{N}(0, \sigma^2)\), \(\alpha\)的估计量\(U\)和\(\beta\)的估计量\(V\)是随机变量\(Y_1, ..., Y_n\)的函数, 是随机变量.

在所有这些估计量中, 考虑``线性估计量'', 即 
$$
U = u_0 + \sum_{i=1}^{n}u_i Y_i, \quad V = v_0 + \sum_{i=1}^{n}v_i Y_i.
$$
其中\(u_0, u_1, ..., u_n, v_0, v_1, ..., v_n\)是常数. 称\(U, V\)是无偏的, 如果\(\mathbb{E}\{U\} = \alpha, \mathbb{E}\{V\} = \beta\).
由于\(\mathbb{E}\{\varepsilon_i\} = 0\), 则无偏性蕴含:
$$
\begin{gathered}
\alpha = \mathbb{E}(U) = u_0 + \sum_{i=1}^{n}u_i(\alpha + \beta x_i) = u_0 + \alpha\left(\sum_{i=1}^{n}u_i \right) + \beta \left(\sum_{i=1}^{n}u_i x_i\right), \\
\beta = \mathbb{E}(V) = v_0 + \sum_{i=1}^{n}v_i(\alpha + \beta x_i) = v_0 + \alpha \left(\sum_{i=1}^{n}v_i \right) + \beta \left(\sum_{i=1}^{n}v_i x_i\right), \\
\end{gathered}
$$
上述等式的成立应该对任意的\(\alpha, \beta\)都成立. 从而我们有
$$
\begin{aligned}
u_0 & =0 \quad \sum_{i=1}^n u_i=1 \quad \text { and } \quad \sum_{i=1}^n u_i x_i=0 \\
v_0 & =0 \quad \sum_{i=1}^n v_i=0 \quad \text { and } \quad \sum_{i=1}^n v_i x_i=1
\end{aligned}
$$
可以证明: 在均方误差最小意义下, 解为 
$$
v_i = \frac{x_i - \bar{x}}{\sum_{j=1}^{n}(x_j - \bar{x})^2}, \quad u_i = \frac{1}{n} - \bar{x}v_i.
$$
\end{framed}


% \medskip

% \printbibliography


\end{document}