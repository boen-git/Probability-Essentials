\documentclass[UTF8, a4paper]{article}
\usepackage{ctex}
\usepackage{graphicx}
\usepackage[margin=2.5cm]{geometry}
\usepackage{subcaption}
\usepackage{amssymb}
\usepackage{amsthm}
\usepackage{amsmath}
\usepackage{enumerate}
\usepackage[backend=bibtex, style=alphabetic]{biblatex}
\usepackage{framed}
\usepackage{mathrsfs} 
\usepackage{xcolor}
\newcommand{\Perp}{\perp\!\!\!\!\perp}
\newtheorem{exercise}{Exercise \#18.}
\newtheorem*{proposition}{命题}
\newtheorem*{remark}{注}
\everymath{\displaystyle}

\addbibresource{my.bib}
\title{Chapter 18: 弱收敛}
\author{}
\date{Latest Update: \today}
\begin{document}
\maketitle

\begin{framed}
弱收敛存在连续映射定理.
若\(X_n \xrightarrow{\mathcal{D}} X\), 
则
$$
\mathbb{E}\{f(X_n)\} \xrightarrow{n\to\infty} \mathbb{E}\{f(X)\}, \quad \forall f \in \mathscr{C}_b(\mathbb{R}).
$$
只需证:
$$
\mathbb{E}\{f(g(X_n))\} \xrightarrow{n\to\infty} \mathbb{E}\{f(g(X))\}, \quad \forall f \in \mathscr{C}_b(\mathbb{R}).
$$
当\(g\)是连续函数时, 有\(F = f\circ g\)是有界连续函数. 根据定义即得连续映射定理.
\end{framed}



\begin{framed}
\begin{exercise}
证明: 若$X_n \xrightarrow{L^p} X(p \geq 1)$, 则$X_n \xrightarrow{D} X$.
\end{exercise}
\end{framed}

\begin{proof}
根据Liapunov不等式, 对\(\forall p \geq 1\), 有
$$
\mathbb{E}\{|X|\} \leq \left[\mathbb{E}\{|X|^p\}^{\frac{1}{p}}\right].
$$
由于\(X_n \xrightarrow{L^p} X\), 所以\(\mathbb{E}\{|X_n - X|^p\} \to 0\).

根据定理18.7, 考察\(\forall f \in \mathscr{C}_{\mathrm{b,Lip}}(\mathbb{R})\), 
存在常数\(K\)使得
\begin{align*}
    |\mathbb{E}\{f(X_n)\} - \mathbb{E}\{f(X)\}| &\leq K\mathbb{E}\{|X_n - X|\} \\
    &\leq K\left[\mathbb{E}\{|X_n - X|^p\}^{\frac{1}{p}}\right] \\
    & \to 0 \,\,(n\to \infty).
\end{align*}

\end{proof}





\begin{framed}
\begin{exercise}
设\(\alpha \in \mathbb{R}^d\). 用构造方法证明: 存在一个连续函数\(f: \mathbb{R}^d \to \mathbb{R}\), 使得
\(0 \leq f(x) \leq 1, \forall x \in \mathbb{R}^d\), \(f(\alpha) = 0\)且对给定的\(\varepsilon >0\), 当满足\(\|x - \alpha\| \geq \varepsilon\)时, 有\(f(x) = 1\).
\end{exercise}
\end{framed}
\begin{remark}
首先在\(d = 1\)时解决这个问题, 再模仿证明到\(d > 1\)的情况.
\end{remark}

\begin{proof}
只需取
$$
f(x) = \max \left\{\frac{1}{\varepsilon} \|x - \alpha\|,  1\right\}.
$$
\end{proof}


\begin{framed}
\begin{exercise}
设\(X\)是一个实值随机变量, 分布函数为\(F\). 证明: \(F(x-) = F(x)\)当且仅当\(P(X = x) = 0\).
\end{exercise}
\end{framed}
\begin{proof}
    根据引理7.1, 有\(P(X = x) = F(x) - F(x-)\).
\end{proof}

\begin{framed}
\begin{exercise}
设\(g: \mathbb{R} \to \mathbb{R}, 0 \leq g(\alpha) \leq 1\), \(g\)是非增的, 并假设\(g\)是右连续的(即\(\lim_{y \downarrow x}g(y) = g(x), \forall x \)).
证明: \(g\)处处有左极限(即\(\lim_{y \uparrow x}g(y) = g(x-)\)存在, \(\forall x\)), 并且集合\(\Lambda = \{x: g(x-) \neq g(x)\}\)是至多可列集.
\end{exercise}
\end{framed}

\begin{remark}
首先证明: 仅有有限的\(x\)使得\(g(x) - g(x-) > \frac{1}{k}\), 然后令\(k\)趋于无穷. 在习题7.16中已经证过.
\end{remark}

\begin{proof}
首先, 
$$
\Lambda = \{x: g(x-) \neq g(x)\} = \{x: g(x-) < g(x)\} = \bigcup_{k=1}^{\infty}\left\{x: g(x) - g(x-) > \frac{1}{k}\right\}.
$$
由于\(g \in [0,1]\)
$$
\sharp \left\{x: g(x) - g(x-) > \frac{1}{k}\right\} \leq k.
$$
仅有有限的\(x\)使得\(g(x) - g(x-) > \frac{1}{k}\), 从而\(\Lambda\)是至多可列的.

接下来证明\(g\)处处有左极限. 令\(t \in \mathbb{R}^1\), 考察\(M = \{g(x): x < t\}\).
则\(M\)非空, 且有上界\(g(t)\), 于是根据确界存在定理, 存在\(\alpha = \sup M\).
根据上确界的定义, 对\(\forall \varepsilon > 0\), 存在\(x_0 < t\)使得\(\alpha - \varepsilon < g(x) \leq \alpha\).
取\(\delta = \alpha - x_0\), 则当\(x \in (t - \delta, t)\)时, \(M-\varepsilon < g(x) \leq M\).
从而\(\lim_{y \uparrow t}g(y) = \alpha = g(t-)\).
\end{proof}


\begin{framed}
\begin{exercise}
设\(F\)是一个实值随机变量的分布函数. 令\(D = \{x : F(x - ) = F(x)\}\).
证明: \(D\)在\(\mathbb{R}\)上是稠密的.
\end{exercise}
\end{framed}
\begin{remark}
可以用18.4证明\(D\)的补集是至多可列的.
\end{remark}

\begin{proof}
根据18.4, 令\(D = \{x : F(x - ) = F(x)\}\), 则\(D^c = \{x : F(x - ) < F(x)\}\)是至多可列的.
反证法, 若不然, 则存在\(x_0 \in D^c, \delta > 0\)使得\(B(x_0, \delta) \subset D^c\). 而这与\(D^c\)是至多可列的矛盾.

\end{proof}


\begin{framed}
\begin{exercise}
设\(\{X_n\}_{n\geq 1 }\)是一列实值随机变量, 且\(X_n \sim Unif[-n,n]\).
在何种意义下, 有\(X_n\)收敛到某个随机变量\(X\)?
\end{exercise}
\end{framed}

\begin{proof}
由于依分布收敛的收敛性是最弱的. 因此先考察依分布收敛.
若存在一个随机变量\(X\)使得\(X_n \xrightarrow{D} X\),
则根据依分布收敛的定义, 记\(F_n\)和\(F\)分别是\(X_n\)和\(X\)的分布函数, 则对于\(F\)的连续点\(x\), 有
$$
F_n(x) \to F(x) \quad (n \to \infty).
$$
注意到
$$
F_n(x) = P(X_n \leq x) = \int_{-n}^{x} \frac{1}{2n} d t = \frac{x+n}{2n} \mathbb{I}_{(-n, n)}(x) + \mathbb{I}_{[n, \infty)}(x).
$$
则对于\(x \in \mathbb{R}^1\), 当\(n\to\infty\)有\(F_n(x) \to \frac{1}{2}\).
而这与\(\lim_{x\to\infty} F(x) = 1\)矛盾! 因此\(X_n\)不依分布收敛.
\end{proof}


\begin{framed}
\begin{exercise}
设\(f_n(x)\)是\(\mathbb{R}\)上的密度函数. 假设\(\lim_{n\to\infty} f_n(x) = e^{-x}\mathbb{I}_{(X>0)}\).
若\(f_n\)是随机变量\(X_n\)的密度. 考察当\(n\)趋于无穷时, \(X_n\)的收敛性.
\end{exercise}
\end{framed}

\begin{proof}
根据定理18.5(Scheff\'{e}定理), \(f_n\)几乎处处逐点收敛于\(f\), 其中\(f(x) = e^{-x}\mathbb{I}_{(X>0)}\)是参数为1的指数分布的密度函数.
则\(X_n \xrightarrow{\mathcal{D}} X\).
\end{proof}


\begin{framed}
\begin{exercise}
设\(\{X_n\}_{n\geq 1}\)是独立同分布的Cauchy(0,1)随机变量.
令\(Y_n = \frac{X_1 + ... + X_n}{n}\). 证明: \(Y_n\)依分布收敛, 并求出极限分布. \(Y_n\)是否也依概率收敛?
\end{exercise}
\end{framed}

\begin{proof}
在习题17.7中, 已经证明\(Y_n\)也服从Cauchy(0,1)分布. 
显然, \(Y_n\)的分布函数逐点收敛于Cauchy(0,1)的分布函数. 于是, \(Y_n\)依分布收敛于Cauchy(0,1)分布.

\(Y_n\)不依概率收敛. 反证法, 若不然, 存在\(Y\)使得\(Y_n \xrightarrow{p} Y\). 
由于
$$
\{|Y_n -Y_m| > \varepsilon\} \subset \{|Y_n - Y| + |Y_m - Y| > \varepsilon\} \subset \{|Y_n - Y| > \varepsilon/2\} \cup \{|Y_m - Y| > \varepsilon/2\}.
$$
如果\(Y_n \xrightarrow{p} Y\), 则应有\(\lim_{\substack{n \to \infty \\ m \to \infty}} P(|Y_n - Y_m| > \varepsilon) = 0(\forall \varepsilon > 0)\).
但是, 若取\(m=2n\), 则有
$$
\left|Y_n-Y_m\right|=\frac{1}{2}\left|\frac{1}{n} \sum_{i=1}^n X_i-\frac{1}{n} \sum_{i=n+1}^{2 n} X_i\right|
$$
但是此时, \(\frac{1}{n} \sum_{i=1}^n X_i, \frac{1}{n} \sum_{i=n+1}^{2 n} X_i\)独立同Cauchy(0,1)分布, 从而\(\left|Y_n-Y_m\right|\)的分布于\(n\)无关.
即\(\lim_{n\to \infty} P(|Y_{n} - Y_{2n}| > \varepsilon) = P(|U-V| > 2\varepsilon) > 0\), 其中\(U,V\)是独立同分布的Cauchy(0,1)随机变量.
而这产生了矛盾. 
因此, \(Y_n\)不依概率收敛. 
\end{proof}


\begin{framed}
\begin{exercise}
设\(\{X_n\}_{n\geq 1}\)是一列随机变量, 且\(\sup_{n} \mathbb{E}\{X^2\} < \infty\).
设\(\mu_n\)是\(X_n\)诱导的分布测度. 证明: 序列\(\mu_n\)是紧的.
\end{exercise}
\end{framed}

\begin{remark}
可以用Chebyshev不等式.
\end{remark}

\begin{proof}
用Chebyshev不等式, 对任意的\(m > 0\), 有
$$
P\{|X_n| > m\} \leq \frac{1}{m^2}\mathbb{E}\{X_n^2\} .
$$
于是, 
$$
\sup_{n}\mu_n([-m, m]^c) = \sup_{n}P\{|X_n| > m\}\leq \frac{1}{m^2}\sup_{n}\mathbb{E}\{X^2\} < \infty.
$$
即 
$$
\lim_{m\to\infty}\sup_{n}\mu_n([-m, m]^c) =0.
$$
从而, \(\mu_n\)是紧的.
\end{proof}


\begin{framed}
\begin{exercise}
设\(X_n, X, Y\)是实值随机变量, 定义在同一个样本空间\((\Omega, \mathcal{A}, P)\)上.
假设
$$
\lim _{n \rightarrow \infty} \mathbb{E}\left\{f\left(X_n\right) g(Y)\right\}=\mathbb{E}\{f(X) g(Y)\}
$$
当\(f,g\)有界, \(f\)连续, \(g\)是Borel可测时成立.
证明: \((X_n, Y)\)依分布收敛到\((X, Y)\).
进一步假设有Borel可测函数\(h\)满足\(X = h(Y)\), 证明: \(X_n \xrightarrow{P} X\).
\end{exercise}
\end{framed}

\begin{remark}
    注意, \(X_n \xrightarrow{\mathcal{D}} X, Y_n \xrightarrow{\mathcal{D}} Y\) 不能推出 \((X_n, Y_n) \xrightarrow{\mathcal{D}} (X, Y)\).
以下是一个反例:
设\(X_n \sim \mathcal{N}(0, 1), Y_n  = (-1)^n X_n\), 则当\(n\)为奇数时, \((X_n, Y_n)\)依分布收敛于\((X,X)\);
当\(n\)为偶数时, \((X_n, Y_n)\)依分布收敛于\((X, -X)\), 这是两个不同的分布(因为分布函数不同).
因此, 不存在\((X,Y)\)使得\((X_n, Y_n)\)依分布收敛于\((X, Y)\).

此外, 依分布收敛的不同之处在于不满足一些运算法则, 只满足Slutsky定理.
比如, 
$$
X_n \xrightarrow{\mathcal{D}} X, Y_n \xrightarrow{\mathcal{D}} Y \nRightarrow X_n + Y_n \xrightarrow{\mathcal{D}} X + Y.
$$
反例是\(X_n = Z, Y_n = -Z\), 其中\(Z\)是标准正态分布.
\end{remark}

\begin{proof}
称\((X_n, Y)\)依分布收敛到\((X, Y)\)是指 
$$
\mathbb{E}\{\phi(X_n, Y)\} \xrightarrow{n\to\infty} \mathbb{E}\{\phi(X,Y)\},
$$
其中\(\phi \in \mathscr{C}_b(\mathbb{R}^2)\).

% 先取\(g \equiv 1\), 
% $$
% \lim _{n \rightarrow \infty} \mathbb{E}\left\{f\left(X_n\right) \right\}=\mathbb{E}\{f(X) \},
% $$
% 则有\(X_n \xrightarrow{\mathcal{D}} X\).

% 取\(f(x) = e^{it_1 x} = \cos(t_1 x) + i\sin(t_1 x), g(x) = e^{it_2x} = \cos(t_2 x) + i\sin(t_2x)\), 
考察\((X_n, Y)\)的特征函数, 
$$
\begin{aligned}
    \mathbb{E}\{e^{it_1 X_n}e^{it_2 Y}\} &= \mathbb{E}\left[\{\cos(t_1 X) + i \sin(t_1 X)\} \{\cos(t_2 Y) + i \sin(t_2 Y)\}\right] \\
    &= \mathbb{E}\{\cos(t_1 X)\cos(t_2 Y) - \sin(t_1 X)\sin(t_2 Y)\} + i\mathbb{E}\{\sin(t_1 X)\cos(t_2 Y) + \cos(t_1 X)\sin(t_2 Y)\} \\
\end{aligned}
$$
显然, \(\sin(t_1 x), \cos(t_1 x), \sin(t_2 y), \cos(t_2 y)\)都是有界函数, 且连续. 
根据题中的条件, 有
$$
\mathbb{E}\{e^{it_1 X_n}e^{it_2 Y}\} \xrightarrow{n\to\infty} \mathbb{E}\{e^{it_1 X}e^{it_2 Y}\}.
$$
根据Le\'{v}y连续定理, 有\((X_n, Y) \xrightarrow{\mathcal{D}} (X, Y)\).

% 对\(\phi \in \mathscr{C}_b(\mathbb{R}^2)\), 
% 取\(\psi_y(x) := \phi(x, y) \in \mathscr{C}_b(\mathbb{R}^1)\), 则根据定义
% $$
% \lim_{n\to\infty} \mathbb{E}\{\psi_y(X_n)\} = \mathbb{E}\{\psi_y(X)\}.
% $$
% 最后, 取\(y = Y\), 有
% $$
% \mathbb{E}\{\phi(X_n, Y)\} \xrightarrow{n\to\infty} \mathbb{E}\{\phi(X,Y)\}.
% $$


根据\((X_n, Y)\)依分布收敛于\((X,Y)\), \(h\)是Borel可测的, 则根据题中的条件,
$$
(X_n, h(Y)) \xrightarrow{\mathcal{D}} (X, h(Y)).
$$
应用Slutsky定理(习题18.15), 有
$$
X_n - X \xrightarrow{\mathcal{D}} X- X = 0 \quad \Rightarrow \quad X_n - X\xrightarrow{\mathcal{D}} 0 \quad \Rightarrow \quad X_n \xrightarrow{P} X.
$$
\end{proof}


\begin{framed}
\begin{exercise}
设\(\mu_\alpha\)表示参数为\(\alpha\)的Pareto分布(Zeta分布). 设\(\alpha_n \to \alpha\), 证明: \(\mu_{\alpha_n}\)弱收敛于\(\mu_{\alpha}\).
\end{exercise}
\end{framed}

\begin{proof}
Zeta分布的pmf是
$$
P(X = j) = c \frac{1}{j^{\alpha + 1}}, j = 1,2,3,...
$$
其中\(\alpha > 0, c = \frac{1}{\zeta(\alpha + 1)}\), \(\zeta\)是Riemann zeta函数.
则 
$$
P(X_n = j) = c_n \frac{1}{j^{\alpha_n + 1}}, j = 1,2,3,...
$$
其中\(\alpha_n > 0, c_n = \frac{1}{\zeta(\alpha_n + 1)}\).

从而
$$
P(X_n = j) \xrightarrow{n\to\infty} P(X = j), \forall j = 1,2,3,...
$$
根据可列取值情况下的弱收敛, \(\mu_{\alpha_n}\)弱收敛于\(\mu_{\alpha}\).
\end{proof}



\begin{framed}
    \begin{exercise}
设\(\mu_\alpha\)表示参数为\(\alpha\)的几何分布. 设\(\alpha_n \to \alpha > 0\), 证明: \(\mu_{\alpha_n}\)弱收敛于\(\mu_{\alpha}\).
\end{exercise}
\end{framed}

\begin{proof}
几何分布的pmf是
$$
P(X = k) = (1 - p)^k p, k = 0,1,2,...
$$
其中, \(p \in (0,1)\).

则
$$
P(X_n = k) = (1 - p_n)^k p_n, k = 0,1,2,...
$$
其中, \(p_n \in (0, 1)\).

从而
$$
P(X_n = k) \xrightarrow{n\to\infty} P(X = k), \forall k = 0,1,2,...
$$
根据可列取值情况下的弱收敛, \(\mu_{\alpha_n}\)弱收敛于\(\mu_{\alpha}\).
\end{proof}


\begin{framed}
\begin{exercise}
设\(\mu_{(N,b,n)}\)是一个超几何分布. 假设在\(N\)趋于无穷时, \(p = \frac{b}{N}\)保持是一个常数.
参数\(n\)固定. 证明当\(N\)趋于无穷时, \(\mu_{(N,b,n)}\)弱收敛到一个二项分布\(B(p,n)\).
\end{exercise}
\end{framed}

\begin{proof}
超几何分布的pmf是
$$
P(X = x) = \frac{\binom{b}{x}\binom{N-b}{n-x}}{\binom{N}{n}}, x = 0,1,...,n.
$$
令\(p = \frac{b}{N}\), 则\(b = pN\), \(N-b = (1-p)N\).
注意到, 当\(N\)充分大时, 
$$
\binom{N}{n} \approx \frac{N^n}{n!}
$$
于是
$$
\begin{aligned}
P(X = x) &= \frac{\binom{pN}{x}\binom{(1-p)N}{n-x}}{\binom{N}{n}} \\
&= \frac{\frac{(pN)!}{x!(pN-x)!}\frac{((1-p)N)!}{(n-x)!((1-p)N-(n-x))!}}{\frac{N!}{n!(N-n)!}} \\
&\xrightarrow{N\to\infty} \frac{\frac{(pN)^x}{x!}\frac{((1-p)N)^{n-x}}{(n-x)!}}{\frac{N^n}{n!}} \\
&= \binom{n}{x}p^x(1-p)^{n-x}.
\end{aligned}
$$
根据可列取值情况下的弱收敛, \(\mu_{(N,b,n)}\)弱收敛于\(\mu_{(p,n)}\).
\end{proof}
 


\begin{framed}
\begin{exercise}[Slusky定理]
设\(X_n\)依分布收敛于\(X\), \(Y_n\)依概率收敛于\(c\). 证明: 
\begin{enumerate}[(a)]
\item \(X_nY_n \xrightarrow{D} cX\),
\item \(\frac{X_n}{Y_n} \xrightarrow{D} \frac{X}{c}, (c \neq 0)\).
\end{enumerate}
\end{exercise}
\end{framed}


\begin{proof}
断言:
$$
X_n \xrightarrow{\mathcal{D}} X, Y_n \xrightarrow{P} c \quad \Rightarrow \quad (X_n, Y_n) \xrightarrow{\mathcal{D}} (X, c).
$$
若上述断言成立, 根据连续映射定理, 分别取
$$
g(x,y) = x + y, \quad g(x,y) = xy, \quad g(x,y) = \frac{x}{y},
$$
即可得出
$$
X_n + Y_n \xrightarrow{D} X + c, \quad X_nY_n \xrightarrow{D} cX, \quad \frac{X_n}{Y_n} \xrightarrow{D} \frac{X}{c}.
$$

下证上述断言.
\begin{enumerate}[Step 1]
    \item 断言\((X_n, c) \xrightarrow{\mathcal{D}} (X,c)\). 这等价于$$
    \mathbb{E}\{f(X_n, c)\} \xrightarrow{\mathcal{D}} \mathbb{E}\{f(X, c)\}, \quad \forall f \in \mathscr{C}_b(\mathbb{R}^2).
    $$
    考察一元函数\(g_c(x) := f(x, c)\). 则显然\(g_c \in \mathscr{C}_b(\mathbb{R}^1)\), 根据\(X_n \xrightarrow{\mathcal{D}} X\), 则
    $$
    \mathbb{E}\{g_c(X_n)\} \xrightarrow{n\to\infty} \mathbb{E}\{g_c(X)\}.
    $$
    从而\((X_n, c) \xrightarrow{\mathcal{D}} (X,c)\).
    \item 断言: \(|(X_n, Y_n) - (X_n, c)| \xrightarrow{p} 0\). 事实上, $$
    |(X_n, Y_n) - (X_n, c)| = |Y_n - c| \xrightarrow{p} 0.
    $$
\end{enumerate}
应用定理18.8, 即得证.
\end{proof}


\begin{framed}
\begin{exercise}
设\(\{X_n\}_{n\geq 1}, \{Y_n\}_{n\geq 1}\)是定义在同一个概率空间上的随机变量列, 
设\(X_n \xrightarrow{D}X\)且\(Y_n \xrightarrow{P} 0\). 证明: \(X_n + Y_n\)依分布收敛于\(X\).
\end{exercise}
\end{framed}

\begin{proof}
这是上一个习题的特例.
\end{proof}


\begin{framed}
\begin{exercise}
设实值随机变量列\(\{X_n\}_{n\geq 1}\)有分布函数\(F_n\), 且\(X_n \xrightarrow{D}X\).
设\(p>0\), 证明: 对任意的正数\(N\), 
$$
\int_{-N}^N|x|^p F(d x) \leq \limsup _{n \rightarrow \infty} \int_{-N}^N|x|^p F_n(d x)<\infty .
$$
\end{exercise}
\end{framed}

\begin{remark}
由于\(X_n \xrightarrow{\mathcal{D}} X\), 则取\(f_N(x) = |x|^p \mathbb{I}_{[-N,N]}(x) \in \mathscr{C}_b(\mathbb{R}^1)\), {\color{red}没}有{\color{red}\(\lim_{n\to\infty} \mathbb{E}\{f_N(X_n)\} = \mathbb{E}\{f_N(X)\}\)}. 于是, 
$$
\int_{-N}^N|x|^p F(d x) = \mathbb{E}\{f_N(X)\} \leq \limsup_{n\to\infty} \mathbb{E}\{f_N(X_n)\} = \limsup _{n \rightarrow \infty} \int_{-N}^N|x|^p F_n(d x) \leq N^p < \infty.
$$
\end{remark}

\begin{proof}
上述证明看上去很美好, 但是\(f_N(x) = |x|^p \mathbb{I}_{[-N,N]}(x)\)不是连续函数!
怎么办? 用线性函数连起来!

取
$$
g_{m,N}(x) = \begin{cases}
    |x|^p, & |x| \leq N - \frac{1}{m}, \\
    (|x| - N) \cdot \frac{\left(N - \frac{1}{m}\right)^p}{-\frac{1}{m}}, & N - \frac{1}{m} \leq |x| \leq N, \\
    0, & |x| > N.
\end{cases}
$$
则此时\(g_{m,N}(x) \in \mathscr{C}_b(\mathbb{R}^1)\), 且满足对任意的\(m,N > 0\), 都有 
$$
\lim_{n\to \infty} \mathbb{E}\{g_{m,N}(X_n)\} = \mathbb{E}\{g_{m,N}(X)\}.
$$
而且还有当\(m \to\infty\)时, 有\(g_{m,N}(x) \uparrow f_N(x) = |x|^p \mathbb{I}_{[-N,N]}(x)\), 从而根据单调收敛定理,
$$
\begin{aligned}
    \int_{-N}^{N} |x|^p dF(x) &= \mathbb{E}\{f_N(X)\} = \lim_{m\to\infty} \mathbb{E}\{g_{m,N}(X)\} \quad \forall N. \\
    \int_{-N}^{N} |x|^p dF_n(x) &= \mathbb{E}\{f_N(X_n)\} = \lim_{m\to\infty} \mathbb{E}\{g_{m,N}(X_n)\} \quad \forall N.
\end{aligned}
$$
由于 \(g_{m,N}(x) \leq f_N(x)\), 
$$
\lim_{n\to\infty} \mathbb{E}\{g_{m,N}(X_n)\} \leq \limsup_{n\to\infty} \mathbb{E}\{f_N(X_n)\}.
$$
于是有
$$
\lim_{m\to\infty} \mathbb{E}\{g_{m,N}(X)\} = \lim_{m\to\infty} \lim_{n\to\infty} \mathbb{E}\{g_{m,N}(X_n)\} \leq \limsup_{n\to\infty} \lim_{m\to\infty} \mathbb{E}\{g_{m,N}(X_n)\} = \limsup_{n\to\infty} \mathbb{E}\{f_N(X_n)\}.
$$
从而 
$$
\int_{-N}^{N} |x|^p dF(x) \leq \limsup_{n\to\infty} \int_{-N}^{N} |x|^p dF_n(x) \leq N^p < \infty, \quad \forall N.
$$


\end{proof}


\begin{framed}
\begin{exercise}
设实值随机变量列\(\{X_n\}_{n\geq 1}\)有分布函数\(F_n\), 且\(X\)有分布函数\(F\). 假设对某些\(r >0\), 
$$
\lim _{n \rightarrow \infty} \int_{-\infty}^{\infty}\left|F_n(x)-F(x)\right|^r d x=0
$$
证明: \(X_n \xrightarrow{D}X\).
\end{exercise}
\end{framed}

\begin{remark}
假设存在\(F\)上的连续点\(y\)使得\(\lim_{n\to\infty} F_n(y) \neq F(y)\).
则存在\(\varepsilon > 0\)和一个子列\(\{n_k\}\)使得\(|F_{n_k}(y) - F(y)| > {\varepsilon}, \forall k\).
证明: 对于适当的\(y_1,y_2\), \(|F_{n_k}(x) - F(x)| > \frac{\varepsilon}{2}\), 对要么\(x \in [y_1, y)\), 要么\(x \in (y, y_2]\)成立.
而这产生了矛盾.
\end{remark}

\begin{proof}
这里的证明沿着提示的思路. 用反证法, 若不然, 则存在分布函数的子列使得在某一连续点\(y\)上不收敛.

\begin{itemize}
    \item 若\(F_{n_K}(y) - F(y) < -\varepsilon, \forall k\), 则当\(x \leq y\)时, $$
    F_{n_k}(x) - F(x) \leq F_{n_k}(y) - F(y) + F(y) - F(x) < -\varepsilon + \{F(y) - F(x)\}.
    $$
    由于\(F\)在\(y\)点处连续, 于是, 存在区间\([y_1, y)\)使得\(F(y) - F(x) < \varepsilon/2\), 从而有
    $$
    F_{n_k}(x) - F(x) < -\varepsilon/2, \forall x \in [y_1, y).
    $$
    \item 若\(F_{n_K}(y) - F(y) > \varepsilon, \forall k\), 则当\(x \geq y\)时, $$
    F_{n_k}(x) - F(x) \geq F_{n_k}(y) - F(y) + F(y) - F(x) > \varepsilon + \{F(y) - F(x)\}.
    $$
    由于\(F\)在\(y\)点处连续, 于是, 存在区间\((y, y_2]\)使得\(F(y) - F(x) < \varepsilon/2\), 从而有
    $$
    F_{n_k}(x) - F(x) > \varepsilon/2, \forall x \in (y, y_2].
    $$
\end{itemize}
上述两种情况都蕴含着
$$
\int_{-\infty}^{\infty} |F_{n_k}(x) - F(x)|^r dx \geq |y_1-y|\frac{\varepsilon}{2} > 0.
$$
这与题设矛盾. 因此, \(X_n \xrightarrow{D}X\).
\end{proof}






\begin{framed}
\begin{exercise}[Polya一致收敛定理]
假设\(\{F_n\}_{n\geq 1}\)是一列\(\mathbb{R}\)上的分布函数, 且它们收敛到\(\mathbb{R}\)上的连续的分布函数\(F\).
证明: 这样的收敛关于\(x\)是一致收敛.
\end{exercise}
\end{framed}

\begin{remark}
首先证明存在点\(x_1, ..., x_m\)使得\(F(x_1) < \varepsilon, F(x_{j+1}) - F(x_{j}) < \varepsilon\), 以及\(1 - F(x_m) < \varepsilon\).
接下来说明: 存在\(N\)使得当\(n>N\)时, \(|F_n(x_j) - F(x_j)| < \varepsilon, 1 \leq j \leq m\).
\end{remark}

\begin{proof}
对每个固定的正整数 $k$ , 考虑以下的序列
$$
-\infty=x_{0, k} \leq x_{1, k} \leq \cdots \leq x_{j, k} \leq \cdots \leq x_{k, k}=\infty
$$
其中 $x_{j, k}=\inf \left\{x: F(x) \geq \frac{j}{k}\right\}$ . 
于是根据 $F$ 的连续性, $F\left(x_{j, k}\right) \leq \frac{j}{k}$ , 因此
$$
0 \leq F\left(x_{j+1, k}\right)-F\left(x_{j, k}\right) \leq \frac{1}{k}, \quad 0 \leq j<k
$$
于是对于任给的 $x$ , 存在 $j$ 使得 $x \in\left(x_{j, k}, x_{j+1, k}\right]$ 满足
$$
F_n(x)-F(x) \leq F_n\left(x_{j+1, k}\right)-F\left(x_{j, k}\right) \leq F_n\left(x_{j+1, k}\right)-F\left(x_{j+1, k}\right)+\frac{1}{k}
$$
和
$$
F_n(x)-F(x) \geq F_n\left(x_{j, k}\right)-F\left(x_{j+1, k}\right) \geq F_n\left(x_{j, k}\right)-F\left(x_{j, k}\right)-\frac{1}{k}
$$
于是
$$
\sup _{x \in\left(x_{j, k}, x_{j+1, k}\right]}\left|F_n(x)-F(x)\right| \leq\left|F_n\left(x_{j+1, k}\right)-F\left(x_{j+1, k}\right)\right|+\left|F_n\left(x_{j, k}\right)-F\left(x_{j, k}\right)\right|+\frac{1}{k}
$$
因此
$$
\sup _x\left|F_n(x)-F(x)\right| \leq \max _{1 \leq j \leq k}\left|F_n\left(x_{j, k}\right)-F\left(x_{j, k}\right)\right|+\max _{0 \leq j \leq k}\left|F_n\left(x_{j, k}\right)-F\left(x_{j, k}\right)\right|+\frac{2}{k}
$$
先固定 $k$ 对 $n$ 取极限, 再对 $k$ 取极限即可。
\end{proof}




\begin{framed}
\begin{exercise}
设\(f\)是一致连续函数. \(X,Y\)是两个实值随机变量.
假设当\(|x-y| < \delta\)时, 有\(|f(x) - f(y)| < \varepsilon\). 证明:
$$
|\mathbb{E}\{f(X)\}-\mathbb{E}\{f(X+Y)\}| \leq \varepsilon+2 \sup _x|f(x)| P\{|Y| \geq \delta\} .
$$
\end{exercise}
\end{framed}

\begin{proof}
设\(Z = (X,Y)\)诱导的测度是\(\mu\), 则
$$
\begin{aligned}
    \left|\int f(x) d\mu - \int f(x + y) d\mu\right| &\leq \left(\int_{|y| < \delta} + \int_{|y| \geq \delta}\right) |f(x) - f(x+y)| d\mu \\
    &\leq \varepsilon  + 2\sup_x |f(x)| \mu\{|y| \geq \delta\} \\
    &= \varepsilon + 2\sup_x |f(x)| P\{|Y| \geq \delta\}.
\end{aligned}
$$
\end{proof}


\begin{framed}
\begin{exercise}
设\(\{X_n\}_{n \geq 1}, X, Y\)是实值随机变量, 定义在同一个概率空间上.
假设\(X_n + \sigma Y\)依分布收敛于\(X + \sigma Y\), 对于任意的\(\sigma > 0\).
证明: \(X_n\)依分布收敛于\(X\).
\end{exercise}
\end{framed}
\begin{remark}
可以用18.19的结论.
\end{remark}

\begin{proof}
\(\forall f\in \mathscr{C}_{b,Lip}(\mathbb{R}^1), \sigma > 0\), 
$$
\begin{aligned}
    &\left|\mathbb{E}\{f(X_n)\} - \mathbb{E}\{f(X)\}\right| \\ = &\left|\mathbb{E}\{f(X_n)\} - \mathbb{E}\{f(X_n + \sigma Y)\} + \mathbb{E}\{f(X_n + \sigma Y)\} - \mathbb{E}\{f(X + \sigma Y)\} + \mathbb{E}\{f(X + \sigma Y)\} - \mathbb{E}\{f(X)\}\right| \\
    \leq & I_1 + I_2 + I_3.
\end{aligned}
$$
其中,
$$
\begin{aligned}
    I_1 &= \left|\mathbb{E}\{f(X_n)\} - \mathbb{E}\{f(X_n + \sigma Y)\}\right|, \\
    I_2 &= \left|\mathbb{E}\{f(X_n + \sigma Y)\} - \mathbb{E}\{f(X + \sigma Y)\}\right|, \\
    I_3 &= \left|\mathbb{E}\{f(X + \sigma Y)\} - \mathbb{E}\{f(X)\}\right|.
\end{aligned}
$$
首先, 根据题目中弱收敛的条件, 
$$
I_2 \xrightarrow{n\to\infty} 0, \quad \forall \sigma > 0.
$$
对于\(I_1\), 由于\(f\)是Lipschitz连续的, \(\forall \varepsilon > 0\),
$$
I_1 \leq \varepsilon + 2\sup_x |f(x)| P\{|\sigma Y| \geq \delta\}, \quad \forall n, \sigma > 0.
$$
先令\(n\to\infty\), 再令\(\sigma \to 0\), 有\(I_1 \to 0 \), 同理, \(I_3 \to 0\).

综上, 
$$
\lim_{n\to\infty} \left|\mathbb{E}\{f(X_n)\} - \mathbb{E}\{f(X)\}\right| = 0, \quad \forall f\in \mathscr{C}_{b,Lip}(\mathbb{R}^1).
$$
这表明\(X_n\)依分布收敛于\(X\).
\end{proof}


\begin{framed}
\begin{exercise}
设\(X\)和\(Y\)是定义在同一个概率空间上的实值独立随机变量. 
假设\(Y\)服从标准正态分布, 令\(f\)是有界连续函数. 证明:
$$
\mathbb{E}\{f(X+\sigma Y)\}=\mathbb{E}\left\{f_\sigma(X)\right\}
$$
其中, 
$$
f_\sigma(x)=\frac{1}{\sqrt{2 \pi} \sigma} \int_{-\infty}^{\infty} f(z) e^{-\frac{1}{2}|z-x|^2 / \sigma^2} d z
$$
证明\(f_\sigma(x)\)是有界的, 且是任意阶可导的.
\end{exercise}
\end{framed}

\begin{proof}
设\(Z = (X,Y)\)的分布函数是\(P^{Z}\), \(X,Y\)的分布函数分别是\(P^X, P^Y\).
则根据独立性(Cor 10.1), 
$$
P^{Z} = P^X \otimes P^Y.
$$
于是根据\(f\)有界, 应用Fubini定理,
$$
\begin{aligned}
    \mathbb{E}\{f(X+\sigma Y)\} &= \iint f(x+\sigma y) dP^{Z}(x,y) \\
    &= \iint f(x+\sigma y) dP^X(x) dP^Y(y) \\
    &= \int \left(\int f(x+\sigma y) dP^Y(y)\right) dP^X(x) \\
    &= \int \int_{-\infty}^{\infty} f(x+\sigma y) \frac{1}{\sqrt{2\pi}} e^{-\frac{1}{2}y^2} dy dP^X(x) \\
    &= \int \int_{-\infty}^{\infty} f(z) \frac{1}{\sqrt{2\pi}\sigma} e^{-\frac{1}{2}|z-x|^2/\sigma^2} dz dP^X(x) \quad \text{(令\(z = x + \sigma y\))} \\
\end{aligned}
$$
于是, 
$$
\mathbb{E}\{f(X+\sigma Y)\} = \mathbb{E}\{f_\sigma(X)\}.
$$

接下来证明\(f_\sigma(x)\)是有界的. 由于\(f\)有界, 存在\(M > 0\)使得\(|f(x)| \leq M\), 于是
$$
|f_\sigma(x)| \leq \frac{1}{\sqrt{2\pi}\sigma} \int_{-\infty}^{\infty} |f(z)| e^{-\frac{1}{2}|z-x|^2/\sigma^2} dz \leq \frac{M}{\sqrt{2\pi}\sigma} \int_{-\infty}^{\infty} e^{-\frac{1}{2}|z-x|^2/\sigma^2} dz = M.
$$

最后证明\(f_\sigma(x)\)是任意阶可导的. 
先考虑一阶导数.
由于 
$$
h'(x) = \lim_{\Delta x \to 0} \frac{h(x+\Delta x) - h(x)}{\Delta x},
$$
为了交换极限和积分, 我们需要找到一个积分内的函数的控制函数.
考察\(\Delta x\)充分小时, 
$$
\left|f(z) \left(e^{-(z - x - \Delta x)^2} - e^{-(z - x )^2}\right)\right| \leq |f(z)| e^{-(z - x)^2} |e^{2z\Delta x + \Delta x^2} - 1| \leq M e^{-(z - x)^2} (2|z| + |\Delta x|) |\Delta x|.
$$
可积, 从而可以交换积分和极限.
\end{proof}



\begin{framed}
\begin{exercise}
设\(\{X_n\}_{n\geq 1}, X\)是实值随机变量. 证明: \(X_n \xrightarrow{D}X\)当且仅当对任意的有界\(\mathcal{C}^\infty\)函数\(f\), 有\(\mathbb{E}\{f(X_n)\} \to \mathbb{E}\{f(X)\}\).
\end{exercise}
\end{framed}

\begin{proof}
只需证反方向. 取(构造)\(Y\)与\(X, \{X_n\}\)独立, 且服从标准正态分布.

若\(\forall f \in \mathscr{C}_b^\infty\), \(\mathbb{E}\{f(X_n)\} \to \mathbb{E}\{f(X)\}\),
特别地, 习题18.21中的函数\(f_\sigma(x)\)是有界的, 且是任意阶可导的.
于是\(\forall \sigma > 0\),
$$
\mathbb{E}\{f(X_n + \sigma Y)\} \xrightarrow{{n\to\infty}} \mathbb{E}\{f(X + \sigma Y)\}.
$$
根据习题18.20, 有\(X_n \xrightarrow{D}X\).
\end{proof}








% \medskip

% \printbibliography


\end{document}