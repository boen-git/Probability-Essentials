\documentclass[UTF8, a4paper]{article}
\usepackage{ctex}
\usepackage{graphicx}
\usepackage[margin=2.5cm]{geometry}
\usepackage{subcaption}
\usepackage{amssymb}
\usepackage{amsthm}
\usepackage{amsmath}
\usepackage{enumerate}
\usepackage[backend=bibtex, style=alphabetic]{biblatex}
\usepackage{framed}
\usepackage{mathrsfs} 
\usepackage{xcolor}
\newcommand{\Perp}{\perp\!\!\!\!\perp}
\newtheorem{exercise}{Exercise \#12.}
\newtheorem*{proposition}{命题}
\newtheorem*{remark}{注}
\everymath{\displaystyle}

\addbibresource{my.bib}
\title{Chapter 12: \(\mathbb{R}^n\)上的概率分布}
\author{}
\date{Latest Update: \today}
\begin{document}
\maketitle

\begin{framed}
\begin{exercise}
证明:
$$
\int_{-\infty}^{\infty} \int_{-\infty}^{\infty} e^{\frac{-\left(x^2+y^2\right)}{2 \sigma^2}} d x d y=2 \pi \sigma^2
$$
因此, $\frac{1}{2 \pi \sigma^2} e^{-\left(x^2+y^2\right) / 2 \sigma^2}$是一个概率密度函数.
\end{exercise}
\end{framed}

\begin{proof}
做三角变换.
令
$$
\begin{cases}
x = r\cos(\theta), \\
y = r\sin(\theta).
\end{cases}
$$
其中, \(r \geq 0\), \(\theta \in [0, 2\pi)\).
于是逆变换为
$$
\begin{cases}
r = \sqrt{x^2+y^2}, \\
\theta = \arctan\left(\frac{y}{x}\right).
\end{cases}
$$
计算雅可比行列式:
$$
\begin{aligned}
J = \frac{\partial(x,y)}{\partial(r,\theta)} &= \begin{vmatrix}
\frac{\partial x}{\partial r} & \frac{\partial y}{\partial r} \\
\frac{\partial x}{\partial \theta} & \frac{\partial y}{\partial \theta}
\end{vmatrix} \\
&= \begin{vmatrix}
\cos(\theta) & \sin(\theta) \\
-r\sin(\theta) & r\cos(\theta)
\end{vmatrix} \\
&= r.
\end{aligned}
$$
于是函数的积分是
$$
\begin{aligned}
    &\int_{-\infty}^{\infty} \int_{-\infty}^{\infty} e^{\frac{-\left(x^2+y^2\right)}{2 \sigma^2}} d x d y \\
    =& \int_{0}^{2\pi} \,d\theta \int_{0}^{\infty} \exp\left(-\frac{r^2}{2\sigma^2}\right) r \,dr \\
    =& 2\pi \int_{0}^{\infty} \exp\left(-\frac{r^2}{2\sigma^2}\right) r \,dr \\
    =& 2\pi \sigma^2.aqaaq
\end{aligned}
$$
从而$\frac{1}{2 \pi \sigma^2} e^{-\left(x^2+y^2\right) / 2 \sigma^2}$是一个概率密度函数.
\end{proof}


\begin{framed}
\begin{exercise}
假设一个联合密度可以分解为: \(f_{(X,Y)}(x, y) = g(x) h(y)\). 求出\(f_X(x)\)和\(f_Y(y)\).
\end{exercise}
\end{framed}

\begin{proof}
根据定义, 
$$
f_X(x) = \int_{-\infty}^{\infty} f_{(X,Y)}(x,y) dy = \int_{-\infty}^{\infty} g(x) h(y) dy = g(x) \int_{-\infty}^{\infty} h(y) \,dy.
$$
同理, \(f_Y(y) = h(y) \int_{-\infty}^{\infty} g(x)\, dx\).
由于
$$
\iint_{\mathbb{R}^2} f_{(X,Y)}(x,y) dxdy = 1,
$$
从而\(\int_{-\infty}^{\infty} g(x)\, dx \int_{-\infty}^{\infty} h(y) \,dy = 1\), 
这表明, 如果\(f_{(X,Y)}(x,y) = g(x) h(y)\), 可以写成因子分解的形式, 
那么\(X, Y\)分别的密度函数是
$$
\begin{gathered}
f_X(x) = \frac{g(x)}{\int_{-\infty}^{\infty} g(x)\, dx} , \\
f_Y(y) = \frac{h(y)}{\int_{-\infty}^{\infty} h(y)\, dy}.
\end{gathered}
$$
还有\(X \Perp Y\).
\end{proof}


\begin{framed}
\begin{exercise}
设\((X,Y)\)有联合密度
$$
\begin{aligned}
& f(x, y)=\frac{1}{2 \pi \sigma_1 \sigma_2 \sqrt{1-r^2}} \\
& \quad \times \exp \left(-\frac{1}{2\left(1-r^2\right)}\left\{\frac{\left(x-\mu_1\right)^2}{\sigma_1^2}-\frac{2 r\left(x-\mu_1\right)\left(y-\mu_2\right)}{\sigma_1 \sigma_2}+\frac{\left(y-\mu_2\right)^2}{\sigma_2^2}\right\}\right)
\end{aligned}
$$
求出\(f_{X=x}(y)\).
\end{exercise}
\end{framed}

\begin{proof}
按目前条件密度的定义, 
$$
f_{X = x}(y) = \frac{f(x, y)}{f_X(x)}.
$$
而
$$
\frac{1}{2\pi\sigma_1\sigma_2 \sqrt{1 - r^2}} = \frac{1}{\sqrt{2\pi\sigma_1^2}} \frac{1}{\sqrt{2\pi \sigma_2^2(1 - r^2)}}
$$
以及
$$
\begin{aligned}
    % \frac{1}{2\pi \sigma_1 \sigma_2 \sqrt{1 - r^2}} 
&\exp\left\{-\frac{1}{2(1 - r^2)}\left[\left(\frac{x - \mu_1}{\sigma_1}\right)^2 - 2r \left(\frac{x - \mu_1}{\sigma_1}\right)\left(\frac{y - \mu_2}{\sigma_2}\right) + \left(\frac{y - \mu_2}{\sigma_2}\right)^2\right]\right\} \\
=&\exp \left\{ - \frac{1}{2(1 - r^2)}\left[(1-r^2)\left(\frac{x - \mu_1}{\sigma_1}\right)^2 + r^2 \left(\frac{x - \mu_1}{\sigma_1}\right)^2 - 2r\left(\frac{x - \mu_1}{\sigma_1}\right)\left(\frac{y - \mu_2}{\sigma_2}\right) + \left(\frac{y - \mu_2}{\sigma_2}\right)^2\right]\right\} \\
=& \exp\left\{-\frac{1}{2\sigma_1^2}(x - \mu_1)^2\right\} \exp\left\{-\frac{1}{2(1 - r^2)}\left[\left(\frac{y - \mu_2}{\sigma_2} \right)- r \left(\frac{x - \mu_1}{\sigma_1}\right)\right]^2\right\} \\
=& \underset{ \frac{1}{\sqrt{2\pi\sigma_1^2}} }{\underbrace{\exp\left\{-\frac{1}{2\sigma_1^2}(x - \mu_1)^2\right\}}} \cdot \underset{\frac{1}{\sqrt{2\pi \sigma_2^2(1 - r^2)}}}{\underbrace{\exp\left\{-\frac{1}{2(1 - r^2)\sigma_2^2}\left[\left({y - \mu_2} \right)- r \frac{\sigma_2}{\sigma_1}\left({x - \mu_1}\right)\right]^2\right\}}}
\end{aligned}
$$

接下来, 我们根据上述分解式计算二元正态的边际密度函数.
$$
\begin{aligned}
    f_X(x) =& \int_{-\infty}^{\infty} f(x, y) \, dy  \\
    =& (2\pi\sigma_1^2)^{\frac{1}{2}}\exp\left\{-\frac{1}{2\sigma_1^2}(x - \mu_1)^2\right\} \int_{-\infty}^{\infty}\frac{1}{\sqrt{2\pi \sigma_2^2(1 - r^2)}} \exp\left\{-\frac{1}{2(1 - r^2)\sigma_2^2}\left[y - \left\{\mu_2 + r \frac{\sigma_2}{\sigma_1}\left({x - \mu_1}\right)\right\}\right]^2\right\} dy \\
    =& (2\pi\sigma_1^2)^{\frac{1}{2}}\exp\left\{-\frac{1}{2\sigma_1^2}(x - \mu_1)^2\right\} 
\end{aligned}
$$
于是
$$
f_{X = x}(y) = \frac{f(x, y)}{f_X(x)} = \frac{1}{\sqrt{2\pi \sigma_2^2(1 - r^2)}} \exp\left\{-\frac{1}{2(1 - r^2)\sigma_2^2}\left[y - \left\{\mu_2 + r \frac{\sigma_2}{\sigma_1}\left({x - \mu_1}\right)\right\}\right]^2\right\}.
$$
服从均值为\(\mu_2 + r \frac{\sigma_2}{\sigma_1}\left({x - \mu_1}\right)\), 方差为\(\sigma_2^2(1 - r^2)\)的正态分布.
\end{proof}

\begin{remark}
上述推导也可以用多元正态分布的推导进行.
$$
\binom{X}{Y} \sim \mathcal{N}_2 \left(\binom{\mu_1}{\mu_2}, \left[\begin{matrix}
    \sigma_1^2 & r\sigma_1\sigma_2 \\
    r\sigma_1\sigma_2 & \sigma_2^2
    \end{matrix}\right]\right)
$$
根据多元正态分布的线性组合还是正态分布, 取\(B = (1 \quad 0)\), 有\(X = B\binom{X}{Y} \sim \mathcal{N}_1 (\mu_1, \sigma_1^2)\).
$$
Y|X \sim \mathcal{N}_1\left(\mu_{2\cdot 1}, \Sigma_{22\cdot 1}\right).
$$
其中
$$
\mu_{2\cdot 1} = \mu_2 + \Sigma_{21} \Sigma_{11}^{-1}(x - \mu_1), \quad \Sigma_{22\cdot 1} = \Sigma_{22} - \Sigma_{21}\Sigma_{11}^{-1}\Sigma_{12}.
$$
\end{remark}



\begin{framed}
\begin{exercise}
令\(\rho_{X,Y}\)表示\((X,Y)\)的相关系数. 设\(a>0,c>0, b\in\mathbb{R}\). 证明:
$$
\rho_{a X+b, c Y+b}=\rho_{X, Y}
$$
这表明, 相关系数不受测量尺度(线性变换)的影响.
\end{exercise}
\end{framed}

\begin{proof}
$$
\begin{aligned}
    \rho_{a X+b, c Y+b} =& \frac{\operatorname{Cov}(aX + b, cY + d)}{\sqrt{\operatorname{Var}(aX + b)\operatorname{Var}(cY + d)}} \\
    =& \frac{ac\operatorname{Cov}(X, Y)}{\sqrt{a^2c^2\operatorname{Var}(X)\operatorname{Var}(Y)}} \\
    =& \frac{\operatorname{Cov}(X, Y)}{\sqrt{\operatorname{Var}(X)\operatorname{Var}(Y)}} \\
    =& \rho_{X, Y}.
\end{aligned}
$$
\end{proof}


\begin{framed}
\begin{exercise}
若\(a\neq 0\), 证明:
$$
\rho_{X, a X+b}=\frac{a}{|a|},
$$
于是, 若\(Y = aX+b\)是\(X\)的非常数仿射变换, 则\(\rho_{X,Y} = \pm 1\).
\end{exercise}
\end{framed}

\begin{proof}
$$
\begin{aligned}
    \rho_{X, aX+b} =& \frac{\operatorname{Cov}(X, aX+b)}{\sqrt{\operatorname{Var}(X)\operatorname{Var}(aX+b)}} \\
    =& \frac{a\operatorname{Var}(X)}{\sqrt{\operatorname{Var}(X)\operatorname{Var}(aX+b)}} \\
    =& \frac{a}{|a|}.
\end{aligned}
$$
\end{proof}


\begin{framed}
\begin{exercise}
设\(X,Y\)方差有限, 令
$$
Z=\left(\frac{1}{\sigma_Y}\right) Y-\left(\frac{\rho_{X, Y}}{\sigma_X}\right) X.
$$
证明: \(\sigma_Z^2 = 1- \rho_{X,Y}^2\), 并推导若\(\rho_{X,Y} = \pm 1\), 则\(Y\)是\(X\)的非常数仿射变换.
\end{exercise}
\end{framed}

\begin{proof}
$$
\begin{aligned}
    \sigma_Z^2 =& \operatorname{Var}\left(\left(\frac{1}{\sigma_Y}\right) Y-\left(\frac{\rho_{X, Y}}{\sigma_X}\right) X\right) \\
    =& \left(\frac{1}{\sigma_Y}\right)^2 \operatorname{Var}(Y) + \left(\frac{\rho_{X, Y}}{\sigma_X}\right)^2 \operatorname{Var}(X) - 2\left(\frac{1}{\sigma_Y}\right)\left(\frac{\rho_{X, Y}}{\sigma_X}\right)\operatorname{Cov}(X, Y) \\
    =& 1 + \rho_{X,Y}^2 - 2\rho_{X,Y} \\
    =& 1 - \rho_{X,Y}^2.
\end{aligned}
$$
若\(\rho_{X,Y} = \pm 1\), 则\(\sigma_Z^2 = 0\), 从而\(Z\)是几处常数\(c\), 
$$
\left(\frac{1}{\sigma_Y}\right) Y-\left(\frac{\rho_{X, Y}}{\sigma_X}\right) X = c
$$
从而
$$
Y = \sigma_Y c + \rho_{X,Y}\frac{\sigma_Y}{\sigma_X} X.
$$
\end{proof}



\begin{framed}
\begin{exercise}
令\((X,Y)\)是单位圆上的均匀分布. 即 
$$
f_{(X, Y)}(x, y)=\left\{\begin{array}{l}
\frac{1}{\pi} \text { if } x^2+y^2 \leq 1 \\
0 \text { if } x^2+y^2>1
\end{array}\right.
$$
求出\(R = \sqrt{X^2+Y^2}\)的分布.
\end{exercise}
\end{framed}

\begin{proof}
    设\(F_R\)为\(R\)的分布函数.
当\(r \leq 0\)时, \(F_R(r) = 0\); 当\(r \geq 1\)时, \(F_R(r) = 1\).
\(\forall r \in (0, 1)\),
$$
\begin{aligned}
    F_R(r) =& P\{R \leq r\} \\
    =& r^2.
\end{aligned}
$$
于是密度函数为
$$
f_R(r) = 2r\mathbb{I}_{(0,1)}(r)
$$
\end{proof}


\begin{framed}
\begin{exercise}
设\((X,Y)\)有密度函数\(f(x,y)\). 求出\(Z = X+Y\)的密度函数.
\end{exercise}
\end{framed}

\begin{proof}
使用卷积公式, 
$$
F_Z(z) = P\{Z \leq z\} = \iint_{x+y \leq z} f(x,y) \,dx\,dy = \int_{-\infty}^{\infty} \int_{-\infty}^{z-x} f(x,y) \,dy\,dx.
$$
令\(y = t-x\), 再运用Fubini定理,
$$
\begin{aligned}
F_Z(z) &= \int_{-\infty}^{\infty} \int_{-\infty}^{z} f(x, t-x) \,dt\,dx \\
&= \int_{-\infty}^{z} \int_{-\infty}^{\infty} f(x, t-x) \,dx\,dt \\
\end{aligned}
$$
根据密度函数的定义,
$$
f_Z(z) = \int_{-\infty}^{\infty} f(x, z-x) \,dx = \int_{-\infty}^{\infty} f(z-y, y) \,dy.
$$
\end{proof}


\begin{framed}
\begin{exercise}
设\(X\)服从\(\mu = 0\), \(\sigma^2 < \infty\)的正态分布. 
设\(\Theta\)服从\((0 ,\pi)\)上的均匀分布, 即密度函数\(f(\theta) = \frac{1}{\pi}\mathbb{I}_{(0,\pi)}(\theta)\).
假设\(X\)和\(\Theta\)是独立的. 求出\(Z = X+a\cos(\Theta)\)的密度函数.
\end{exercise}
\end{framed}

\begin{remark}
这在电子工程中很有用.
\end{remark}

\begin{proof}
\(\Theta\)的密度函数为
$$
f(\theta) = \frac{1}{\pi}\mathbb{I}_{(0,\pi)}(\theta).
$$
令\(Y = a\cos(\Theta)\), 
取\(g(\theta) = a\cos(\theta)\), 在\((0, \pi)\)上是单射, 且\(g^{-1}(y) = \arccos\left(\frac{y}{a}\right)\).
导数为
$$
\frac{d}{dy}\arccos\left(\frac{y}{a}\right) = -\frac{1}{\sqrt{a^2 - y^2}}.
$$
于是\(Y\)的密度函数是
$$
f_Y(y) = f(\arccos\left(\frac{y}{a}\right))\left|\frac{d}{dy}\arccos\left(\frac{y}{a}\right)\right|_+ = \frac{1}{\pi \sqrt{a^2 - y^2}}\mathbb{I}_{(-a, a)}(y).
$$

此外, \(X\)的密度函数是
$$
f_X(x) = \frac{1}{\sqrt{2\pi}\sigma} \exp\left\{-\frac{x^2}{2\sigma^2}\right\}.
$$
根据独立性, 
$$
f_{X,Y}(x,y) = f_X(x)f_Y(y) = \frac{1}{\sqrt{2\pi}\sigma} \exp\left\{-\frac{x^2}{2\sigma^2}\right\} \frac{1}{\pi \sqrt{a^2 - y^2}}\mathbb{I}_{(-a, a)}(y).
$$

根据上一题的卷积公式, 
$$
\begin{aligned}
    f_Z(z) &= \int_{-\infty}^{\infty} f_X(z - y)f_Y(y) \,dy \\
    &= \int_{-\infty}^{\infty} \frac{1}{\sqrt{2\pi}\sigma} \exp\left\{-\frac{(z - y)^2}{2\sigma^2}\right\} \frac{1}{\pi \sqrt{a^2 - y^2}}\mathbb{I}_{(-a, a)}(y) \,dy \\
    &= \frac{1}{\sqrt{2\pi}\sigma} \int_{0}^{\pi} \exp\left\{-\frac{(z - a\cos(w))^2}{2\sigma^2}\right\} \frac{1}{\pi \sqrt{a^2 - a^2 \cos^2(w)}} \,a\sin(w) dw \\
    &= \frac{1}{\pi\sigma \sqrt{2\pi}} \int_{0}^{\pi} \exp\left\{-\frac{(z - a\cos(w))^2}{2\sigma^2}\right\} \,dw.
\end{aligned}
$$



\end{proof}


\begin{framed}
\begin{exercise}
设\(X,Y\)是独立的, 假设\(Z = g(X), W = h(Y)\), 其中\(g\)和\(h\)是单射, 可微函数. 求出\((Z,W)\)的联合密度\(f_{Z,W}(z,w)\).
\end{exercise}
\end{framed}

\begin{proof}
由于\(X,Y\)独立, \(g,h\)可微, 从而可测, 于是\(Z, W\)也是独立的.
$$
\begin{aligned}
    f_{Z,W}(z,w) &= f_Z(z) f_W(w) \\
    &= f_X(g^{-1}(z))f_Y(h^{-1}(w))\left|\frac{\partial g^{-1}}{\partial z}\right|_+\left|\frac{\partial h^{-1}}{\partial w}\right|_+.
\end{aligned}
$$
\end{proof}


\begin{framed}
\begin{exercise}
设\(X,Y\)是独立的, 且都服从\(\mu = 0\), \(\sigma^2 <\infty\)的正态分布. 令
$$
Z=\sqrt{X^2+Y^2} \text { , } W=\operatorname{arctan}\left(\frac{X}{Y}\right), \quad-\frac{\pi}{2}<W \leq \frac{\pi}{2}.
$$
证明\(Z\)服从Rayleigh分布, \(W\)服从\(\left(-\frac{\pi}{2}, \frac{\pi}{2}\right)\)上的均匀分布. \(Z\)和\(W\)独立.
\end{exercise}
\end{framed}

\begin{proof}
由于\((X,Y)\)的联合密度函数为
$$
f_{(X,Y)}(x,y) = \frac{1}{2\pi\sigma^2} \exp\left\{-\frac{x^2+y^2}{2\sigma^2}\right\},
$$
由于
$$
\begin{cases}
Z = \sqrt{X^2+Y^2} \in [0, \infty), \\
W = \arctan\left(\frac{X}{Y}\right) \in \left(-\frac{\pi}{2}, \frac{\pi}{2}\right),
\end{cases}
$$
于是可以分两种情况考虑. 当\(Y > 0\)时, \(Y = Z \cos(W)\); 当\(Y < 0\)时, \(Y = -Z \cos(W)\).
定义
$$
\begin{aligned}
    & S_0 = \{(x,y): y = 0\}, \\
    & S_1 = \{(x,y): y > 0\}, \\
    & S_2 = \{(x,y): y < 0\}.
\end{aligned}
$$
于是\(S_0, S_1, S_2\)构成了对平面\(\mathbb{R}^2\)的划分. 而且\(m_2(S_0) = 0\).
对于\(S_1, S_2\), 
$$
g(x,y) = \left(\sqrt{x^2 + y^2}, \arctan\left(\frac{x}{y}\right)\right)
$$
是单射.(因为是二维的, 只要\(x,y\)中有一个正负号识别出就可以.)
于是在\(S_1\)上, \(g\)可逆, 逆变换为
$$
\begin{cases}
X = Z\sin(W), \\
Y = Z\cos(W).
\end{cases}
$$
Jacobian行列式为
$$
\begin{vmatrix}
\frac{\partial x}{\partial z} & \frac{\partial x}{\partial w} \\
\frac{\partial y}{\partial z} & \frac{\partial y}{\partial w}
\end{vmatrix} = \begin{vmatrix}
    \sin(w) & z\cos(w) \\
    \cos(w) & -z\sin(w)
    \end{vmatrix}  = -z.
$$
在\(S_2\)上, \(g\)可逆, 逆变换为
$$
\begin{cases}
X = Z\sin(W), \\
Y = -Z\cos(W).
\end{cases}
$$
Jacobian行列式为
$$
\begin{vmatrix}
\frac{\partial x}{\partial z} & \frac{\partial x}{\partial w} \\
\frac{\partial y}{\partial z} & \frac{\partial y}{\partial w}
\end{vmatrix} = \begin{vmatrix}
    \sin(w) & z\cos(w) \\
    -\cos(w) & z\sin(w)
    \end{vmatrix}  = z.
$$
从而, \((Z,W)\)的联合密度函数为
$$
\begin{aligned}
    f_{Z, W}(z, w)=&\left(\frac{1}{2 \pi \sigma^2} e^{\frac{-z^2}{2 \sigma}} z+\frac{1}{2 \pi \sigma^2} e^{\frac{-z^2}{2 \sigma}} z\right) \mathbb{I}_{\left(-\frac{\pi}{2}, \frac{\pi}{2}\right)}(w) \mathbb{I}_{(0, \infty)}(z) \\
    =& \frac{z}{\pi\sigma^2} \exp\left\{-\frac{z^2}{2\sigma^2}\right\}\mathbb{I}_{(0, \infty)}(z) \mathbb{I}_{\left(-\frac{\pi}{2}, \frac{\pi}{2}\right)}(w).
\end{aligned}
$$
于是根据习题12.2可得\(Z\)服从Rayleigh分布, \(W\)服从\(\left(-\frac{\pi}{2}, \frac{\pi}{2}\right)\)上的均匀分布. 且\(Z\)和\(W\)独立.

% 于是
% $$
% \begin{cases}
% X = Z\sin(W), \\
% Y = Z\cos(W).
% \end{cases}
% $$
% Jacobian行列式为
% $$
% \begin{vmatrix}
% \frac{\partial x}{\partial z} & \frac{\partial x}{\partial w} \\
% \frac{\partial y}{\partial z} & \frac{\partial y}{\partial w}
% \end{vmatrix} = \begin{vmatrix}
%     \sin(w) & z\cos(w) \\
%     \cos(w) & -z\sin(w)
%     \end{vmatrix}  = -z.
% $$
% 于是
% $$
% f_{Z,W}(z,w) = f_{(X,Y)}(z\sin(w), z\cos(w))\cdot |-z| = \frac{z}{2\pi\sigma^2} \exp\left\{-\frac{z^2}{2\sigma^2}\right\}\mathbb{I}_{(0, \infty)} \mathbb{I}_{\left(-\frac{\pi}{2}, \frac{\pi}{2}\right)}(w).
% $$
% 根据习题12.2可得\(Z\)服从Rayleigh分布, \(W\)服从\(\left(-\frac{\pi}{2}, \frac{\pi}{2}\right)\)上的均匀分布. 且\(Z\)和\(W\)独立.
\end{proof}

\begin{remark}
这一题要仔细分析\(X,Y\)的取值范围, 从而找出对应的{\color{blue}单射区域}.
\end{remark}


\begin{framed}
\begin{exercise}
设\((X_1, ..., X_n)\)是独立随机变量. 定义
$$
\begin{aligned}
& Y_1=\min \left(X_i ; 1 \leq i \leq n\right) \\
& Y_2=\text { second smallest of } X_1, \ldots, X_n \\
& \vdots \\
& Y_n=\text { largest of } X_1, \ldots, X_n.
\end{aligned}
$$
则\(Y_1, ..., Y_n\)也是随机变量, 且有\(Y_1\leq Y_2 \leq ... \leq Y_n \).
因此, \(Y_1, ..., Y_n\)是\(X_1, ..., X_n\)的顺序统计量.
通常记\(Y_k = X_{(k)}\).
假设\(X_i\)独立同分布于相同的密度函数\(f\). 证明顺序统计量的联合密度函数为
$$
f_{\left(X_{(1)}, \ldots, X_{(n)}\right)}\left(y_1, \ldots, y_n\right)= \begin{cases}n!\prod_{i=1}^n f\left(y_i\right) & \text { for } y_1<y_2<\ldots<y_n \\ 0 & \text { otherwise }\end{cases}
$$
\end{exercise}
\end{framed}


\begin{proof}
先证明: 顺序统计量是随机变量.
\(\forall x \in \mathbb{R}\), 定义集合
$$
\begin{aligned}
    A_m(x) =& \{\text{刚好有} m \text{个} X_i \leq x\} \\
           =& \bigcup_{\mathbf{\tau} \in \sigma\{1, ..., n\}}\{X_{\tau_1} \leq x, ..., X_{\tau_m} \leq x, X_{\tau_{m+1}} > x, ..., X_{\tau_n} > x\}.
\end{aligned}
$$
其中\(\tau = (\tau_1, ..., \tau_n)\)表示\(\{1, ..., n\}\)的一个置换.
从而\(A_m(x) \in \mathcal{B}\).
于是, 
$$
\{Y_k \leq x\} = \{X_{(k)} \leq x\} = \bigcup_{m\geq k} A_m(x) \in \mathcal{B}.
$$
因此, 顺序统计量是随机变量.

接下来说明联合分布.

一种方法是按定义. 根据全排列, \(X_1, ..., X_n\)的支集可以分割成\(n!\)个区域. 
这些集合映射到\(Y_1, ..., Y_n\)的支集上, 即
$$
\{(y_1, ..., y_n): a < y_1 < y_2 < ... < y_n < b\}.
$$
对于没种可能的排列映射到\(Y\)上, 根据行列式的定义, 每一个变换的Jacobian只能是\(\pm 1\).
因此, 
$$
\begin{aligned}
g\left(y_1, y_2, \cdots, y_n\right) & =\sum_{i=1}^{n!}\left|J_i\right| f\left(y_1\right) f\left(y_2\right) \cdots f\left(y_n\right) \\
& = \begin{cases}n!f\left(y_1\right) f\left(y_2\right) \cdots f\left(y_n\right), & a<y_1<y_2<\cdots<y_n<b \\
0, & \text { 其他 }\end{cases}
\end{aligned}
$$


一种方法是用概率元方法. 即以\(y_1, y_1 + \Delta y_1, ..., y_n, y_n+\Delta y_n\)为划分, 分成\(2n+1\)堆的多项分布.
$$
\begin{aligned}
g(y_1, ..., y_n) =& \lim_{\substack{\Delta y_1 \to 0 \\ ... \\ \Delta y_n \to 0}} \frac{P\{y_1 < Y_1 \leq y_1 + \Delta y_1, ..., y_n < Y_n \leq y_n + \Delta y_n\}}{\Delta y_1 ... \Delta y_n} \\
=& \lim_{\substack{\Delta y_1 \to 0 \\ ... \\ \Delta y_n \to 0}} \frac{\frac{n!}{0! 1! 0! ... 1! 0!}\{F(y_1)\}^0 \{F(y_1 + \Delta y_1) - F(y_1)\}^1 ... \{F(y_n + \Delta y_n) - F(y_n)\}^1 \{1 - F(y_n + \Delta y_n)\}^0}{\Delta y_1 ... \Delta y_n}\\
=& n! \prod_{i=1}^n f(y_i).
\end{aligned}
$$
\end{proof}




\begin{framed}
\begin{exercise}
令\((X_1, ..., X_n)\)独立同分布于\((0,a)\)的均匀分布.
证明顺序统计量\((X_{(1)}, ..., X_{(n)})\)的联合密度函数为
$$
f\left(y_1, \ldots, y_n\right)= \begin{cases}\frac{n!}{a^n} & \text { for } y_1<y_2<\ldots<y_n \\ 0 & \text { otherwise. }\end{cases}
$$
\end{exercise}
\end{framed}

\begin{proof}
直接应用习题12.12的结论即可.
\end{proof}



\begin{framed}
\begin{exercise}
证明若\(X_1, ..., X_n\)是独立同分布于密度函数\(f\), 分布函数\(F\), 则\(X_{(k)}\)的密度函数是
$$
f_{(k)}(y)=k\binom{n}{k} f(y)(1-F(y))^{n-k} F(y)^{k-1}.
$$
\end{exercise}
\end{framed}

\begin{proof}
证明方法一用定义. 用习题12.12中的记号, 写出分布函数
$$
F_m(x) = P\{Y_k \leq x\} = P \left\{\bigcup_{m\geq k} A_m(x)\right\}
$$
注意到当\(m \neq n\)时, \(A_m(x) \cap A_n(x)= \emptyset, \forall x \in \mathbb{R}\), 从而
$$
\begin{aligned}
    P \left\{\bigcup_{m\geq k} A_m(x)\right\} &= \sum_{m\geq k} P\{A_m(x)\} \\
\end{aligned}
$$
而\(P\{A_m(x)\} = \binom{n}{m} \{F(x)\}^m \{1-F(x)\}^{n-m}\)刚好服从二项分布, 因此
$$
F_m(x) = \sum_{m = k}^{n}\binom{n}{m} \{F(x)\}^m \{1-F(x)\}^{n-m}.
$$
用归纳法可以证明:
$$
\sum_{m = k}^{n} \binom{n}{m} p^m(1-p)^{n-m} = \frac{n!}{(m-1)!(n-m)!}\int_{0}^{p}t^{m-1}(1-t)^{n-m} \, dt.
$$
于是, 
$$
F_m(x) = \frac{n!}{(m-1)!(n-m)!}\int_{0}^{F(x)}t^{m-1}(1-t)^{n-m} \, dt.
$$
对\(F_m(x)\)求导, 即得到\(f_{(k)}(y)\).


另一种证明用概率元方法. 即以\(x, x+\Delta x\)为划分, 分成三堆的多项分布.
$$
\begin{aligned}
    f_{(k)}(y) =& \lim_{\Delta x \to 0}\frac{P(x < X_{(k)} \leq x+ \Delta x)}{\Delta x} \\
    =& \lim_{\Delta x \to 0}\frac{\frac{n!}{(k-1)! 1! (n-k)!} \{F(x)\}^{n-1} \{F(x+\Delta x) - F(x)\}^1 \{1 - F(x + \Delta x)\}^{n-k}}{\Delta x} \\
    =& \frac{n!}{(k-1)! (n-k)!} \{F(x)\}^{n-1}\lim_{\Delta x \to 0} \frac{\{F(x+\Delta x) - F(x)\}}{\Delta x} \{1 - F(x + \Delta x)\}^{n-k} \\
    =& \frac{n!}{(k-1)! (n-k)!} \{F(x)\}^{n-1}  \{1 - F(x)\}^{n-k} f(x).  \\
\end{aligned}
$$
\end{proof}



\begin{framed}
\begin{exercise}[正态随机变量的模拟]
设\(U_1, U_2\)是独立的\((0,1)\)上的均匀分布. 令\(\theta = 2\pi U_1, S = -\ln(U_2)\).
\begin{enumerate}[a)]
    \item 证明\(S\)服从指数分布. \(R = \sqrt{2S}\)服从Rayleigh分布.
    \item 令\(X = R\cos(\theta), Y = R\sin(\theta)\). 证明: \(X,Y\)是独立的, 且都服从\(\mu = 0\), \(\sigma^2 = 1\)的正态分布.
\end{enumerate}
\end{exercise}
\end{framed}
\begin{remark}
这是一种生成正态分布的方法. 被称为Box-Muller变换.
\end{remark}


\begin{proof}
\begin{enumerate}[a)]
    \item \(S = -\ln(U_2)\)的分布函数为
    $$
    F_S(s) = P\{S \leq s\} = P\{-\ln(U_2) \leq s\} = P\{U_2 \geq e^{-s}\} = 1 - e^{-s}.
    $$
    从而\(S\)服从指数分布.
    \(R = \sqrt{2S}\)的分布函数为
    $$
    F_R(r) = P\{R \leq r\} = P\{\sqrt{2S} \leq r\} = P\{2S \leq r^2\} = P\{S \leq \frac{r^2}{2}\} = 1 - e^{-\frac{r^2}{2}}.
    $$
    从而\(R\)服从Rayleigh分布.
    \item 由于\(\theta\)和\(S\)是独立的, 从而\(R\)和\(\theta\)是独立的.
    由于\(R\)服从参数\(\sigma^2 = 1\)的Rayleigh分布, 从而\(X,Y\)的联合密度函数为
    $$
    f_{(X,Y)}(x,y) = \frac{1}{2\pi} e^{-\frac{x^2+y^2}{2}}.
    $$
    于是\(X = R\cos(\theta), Y = R\sin(\theta)\)是独立的, \(X,Y\)都服从\(\mu = 0\), \(\sigma^2 = 1\)的正态分布.
\end{enumerate}
\end{proof}






% \medskip

% \printbibliography


\end{document}