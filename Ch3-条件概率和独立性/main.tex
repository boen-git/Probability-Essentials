\documentclass[UTF8, a4paper]{article}
\usepackage{ctex}
\usepackage{graphicx}
\usepackage[margin=2.5cm]{geometry}
\usepackage{subcaption}
\usepackage{amssymb}
\usepackage{amsthm}
\usepackage{amsmath}
\usepackage{enumerate}
\usepackage{framed}
\newtheorem{exercise}{Exercise \#3.}
\newtheorem*{proposition}{命题}
\newtheorem*{remark}{注记}
\everymath{\displaystyle}
\title{Exercise 3: 条件概率和独立性}
\author{}
\date{Latest Update: \today}
\begin{document}
\maketitle

所有习题中的概率空间都是固定的, 且\(A, B, A_n\)等等皆是事件.

\begin{framed}
\begin{exercise}
    证明当\(A \cap B = \varnothing\), 则除非\(P(A) = 0\)或\(P(B) = 0\), 否则\(A\)和\(B\)不可能独立.
\end{exercise}
\end{framed}

\begin{proof}
    只需证明逆否命题. 若\(A\)和\(B\)独立, 则\(0=P(\varnothing) = P(A \cap B) = P(A)P(B)\), 从而\(P(A) = 0\)或\(P(B) = 0\).
\end{proof}

\begin{framed}
\begin{exercise}
    设\(P(C) > 0\). 证明\(P(A\cup B \mid C) = P(A \mid C) + P(B \mid C) - P(A \cap B \mid C)\).
\end{exercise}
\end{framed}

\begin{proof}
直接按定义展开.
$$
LHS = \frac{P((A\cup B) \cap C)}{P(C)} = \frac{P(A \cap C) + P(B \cap C) - P(A \cap B \cap C)}{P(C)} = RHS.
$$
\end{proof}


\begin{framed}
\begin{exercise}
    若\(P(C) > 0\)且\(A_1, ..., A_n\)两两不交. 证明
$$
P\left(\cup_{i=1}^n A_i \mid C\right)=\sum_{i=1}^n P\left(A_i \mid C\right)
$$
\end{exercise}
\end{framed}
\begin{proof}
首先, 若对\(i\neq j\), \(A_i \cap A_j = \varnothing\), 则\(A_i \cap C\cap A_j \cap C = \varnothing \cap C = \varnothing\). 

由于\(P(C) > 0 \), 根据条件概率的定义, 以及概率的分离可加性, 
$$
P\left(\cup_{i=1}^n A_i \mid C\right) = \frac{P\left(\cup_{i=1}^n A_i \cap C\right)}{P(C)} = \frac{\sum_{i=1}^n P(A_i \cap C)}{P(C)} = \sum_{i=1}^n P(A_i \mid C).
$$ 
\end{proof}


\begin{framed}
\begin{exercise}
    设\(P(B) > 0\). 证明\(P(A \cap B) = P(A \mid B) P(B)\).
\end{exercise}
\end{framed}
\begin{proof}
直接按定义展开. 显然.
$$
P(A \cap B) = \frac{P(A \cap B)}{P(B)}P(B) = P(A \mid B)P(B).
$$
\end{proof}



\begin{framed}
\begin{exercise}
    设\(0 < P(B) < 1\), \(A\)是事件. 证明$$
P(A)=P(A \mid B) P(B)+P\left(A \mid B^c\right) P\left(B^c\right).
$$
\end{exercise}
\end{framed}

\begin{proof}
由于\(B\cap B^c = \varnothing\), \(B \cup B^c = \Omega\), \(0<P(B)<1\), \((A\cap B) \cap (A \cap B^c) = \varnothing\), 因此
$$
P(A)= P(A \cap (B \cup B^c)) = P(A \cap B) + P(A \cap B^c) = P(A \mid B)P(B) + P(A \mid B^c)P(B^c).
$$
\end{proof}


\begin{framed}
\begin{exercise}
    捐献的血液要接受艾滋病筛查. 假设检验的准确率有\(99\%\), 而且在对应年龄的人群中, 艾滋病的患病率是万分之一. 
    同时, 检验存在5\%的假阳性率.
    假设检验结果显示样本是阳性的, 那么该样本患有AIDS的概率是多少? 还是99\%吗?
\end{exercise}
\end{framed}

\begin{remark}
99\%的准确率指的是\(P(\text{阳性} | \text{患有AIDS})\), 目标是求\(P(\text{患有AIDS} \mid \text{阳性})\).
\end{remark}

\begin{proof}
设事件\(A\)表示``检验结果为阳性''这一事件, 事件\(B\)表示``患者患有AIDS''这一事件. 根据题目中的信息, 
$$
P(A \mid B) = 0.99, \quad P(A \mid B^c) = 0.05, \quad P(B) = 0.0001.
$$
于是, 为求出样本阳性下, 该样本患有AIDS的概率, 即求
$$
P(B \mid A) = \frac{P(A \mid B) P(B)}{P(A \mid B)P(B) + P(A \mid B^c) P(B^c)} = \frac{0.99 \times 0.0001}{0.99 \times 0.0001 + 0.05 \times 0.9999} \approx 0.00198.
$$
也就是说, 样本患有AIDS的概率只有不到\(0.2\%\).
\end{proof}


\begin{framed}
\begin{exercise}
    设\(\{A_n\}_{n = 1}^\infty, \{B_n\}_{n = 1}^\infty \subset \mathcal{A}\), 且\(A_n \to A, B_n \to B\), 满足\(P(B) > 0\), \(P(B_n) > 0, \forall n\).
    证明:
    \begin{enumerate}[a)]
        \item \(\lim_{n \to \infty} P(A_n |B) = P(A|B)\),
        \item \(\lim_{n\to \infty}P(A|B_n) = P(A|B)\).
        \item \(\lim_{n\to \infty}P(A_n | B_n) = P(A|B)\).
    \end{enumerate}
\end{exercise}
\end{framed}


\begin{proof}
\begin{enumerate}[a)]
    \item 由于条件概率也是概率, 根据概率测度的连续性定理, 即有\(\lim_{n \to \infty} P(A_n |B) = P(A|B)\).
    \item 由于\(\mathbb{I}_{A\cap B_n}(w) =  \mathbb{I}_A (w)\mathbb{I}_{B_n}(w) \to \mathbb{I}_A(w) \mathbb{I}_B(w) = \mathbb{I}_{A\cap B}(w)\), 于是\(A \cap B_n \to A \cap B, n \to \infty\). 根据概率测度的连续性定理, 即有\(\lim_{n \to \infty} P(A \cap B_n) = P(A \cap B)\), 从而根据数学分析, $$
P\left(A \mid B_n\right)=\frac{P\left(A \cap B_n\right)}{P\left(B_n\right)} \rightarrow \frac{P(A \cap B)}{P(B)}=P(A \mid B) , n\to \infty.
$$
    \item 用示性函数, 同样可以证明\(A_n \cap B_n \to A\cap B(n \to \infty)\), \(\lim_{n\to \infty}P(A_n | B_n) = \lim_{n\to \infty}\frac{P(A_n \cap B_n)}{P(B_n)} = \frac{P(A \cap B)}{P(B)}= P(A|B)\).
\end{enumerate}
\end{proof}


\begin{framed}
\begin{exercise}
    假设我们对有两种结果都投硬币进行建模, 分别用H和T表示正面或背面向上.
    设\(P(H) = P(T) = {1 \over 2}\). 假设现在我们投掷两枚这样的硬币, 使得样本空间\(\Omega\)包含四个点: \(\{HH, HT, TH, TT\}\).
    我们假设投掷是独立的.
    \begin{enumerate}[a)]
        \item 求第一次是正面朝上的条件下两枚硬币都是正面朝上的概率.
        \item 求至少一次正面朝上的条件下两枚硬币都是正面朝上的概率.
    \end{enumerate}
\end{exercise}
\end{framed}

\begin{proof}
\begin{enumerate}[a)]
    \item 设事件\(A\)表示``第一次是正面朝上'', 事件\(B\)表示``两枚硬币都是正面朝上''. 则\(P(A) = \frac{1}{2}\), \(P(B) = \frac{1}{4}\), \(P(A \cap B) = \frac{1}{4}\), 从而\(P(B|A) = \frac{1}{2}\).
    \item 设事件\(C\)表示``至少一次正面朝上'', 则\(P(C) = 1 - P(\text{两次都是背面朝上}) = 1 - \frac{1}{4} = \frac{3}{4}\). 于是, \(P(B|C) = \frac{P(B \cap C)}{P(C)} = \frac{1/4}{3/4} = \frac{1}{3}\).
\end{enumerate}
\end{proof}


\begin{framed}
\begin{exercise}
    假设\(A, B, C\)是独立事件, \(P(A \cap B) \neq 0\). 证明 \(P(C | A\cap B) = P(C)\).
\end{exercise}
\end{framed}

\begin{proof}
$$
LHS = \frac{P(A\cap B\cap C)}{P(A\cap B)} = \frac{P(A)P(B)P(C)}{P(A)P(B)} = RHS.
$$
\end{proof}

\begin{framed}
\begin{exercise}
    一个盒子里有\(r\)个红球, \(b\)个黑球. 从盒子里随机取出一个球, 再在盒子里随机取出一个球, 求出以下事件的概率.
    \begin{enumerate}[a)]
        \item 两个球都是红色的.
        \item 第一个球是红色的, 第二个球是黑色的.
    \end{enumerate}
\end{exercise}
\end{framed}

\begin{proof}
\begin{enumerate}[a)]
    \item 事件\(A\)表示``第一个球是红色的'', 事件\(B\)表示``第二个球是红色的''. 则\(P(A) = \frac{r}{r+b}\), \(P(B \mid A) = \frac{r-1}{r+b-1}\), \(P(A\cap B) = P(A) P(B \mid A)= \frac{r(r-1)}{(r+b)(r+b-1)}\).
    \item 事件\(C\)表示``第一个球是红色的'', 事件\(D\)表示``第二个球是黑色的''. 则\(P(C) = \frac{r}{r+b}\), \(P(D \mid C) = \frac{b}{r+b - 1}\), \(P(C\cap D) = \frac{rb}{(r+b)(r+b-1)}\).
\end{enumerate}
\end{proof}



\begin{framed}
\begin{exercise}[Polya坛子模型]
    一个坛子有\(r\)个红球, \(b\)个蓝球. 从坛子里随机取出一个球, 记录下它的颜色放回, 再额外放进\(d\)个颜色相同的球. 无限重复这个过程. 求以下的概率 
    \begin{enumerate}[a)]
        \item 第二个被拿出的球是蓝色的.
        \item 给定第二个被拿出点球是蓝色的条件下, 第一个球是蓝色的.
    \end{enumerate}
\end{exercise}
\end{framed}

\begin{proof}
\begin{enumerate}[a)]
    \item 设\(B_n\)表示第\(n\)个被拿出的球是蓝色的事件. 则根据全概公式\(P(B_2) = P(B_1) P(B_2 \mid B_1) + P(B_1^c) P(B_2 \mid B_1^c) \), 于是, $$P(B_2) = \frac{b}{r+b} \frac{b+d}{r+b +d} + \frac{r}{r+b}\frac{b}{r+b+d} = \frac{b(r+b+d)}{(r+b)(r+b+d)} = \frac{b}{r+b}.$$
    \item 为计算\(P(B_1 \mid B_2)\), 用Bayes公式, \(P(B_1 \mid B_2) = \frac{P(B_2 \mid B_1)P(B_1)}{P(B_2 \mid B_1)P(B_1) + P(B_2 \mid B_1^c)P(B_1^c)}\), 于是, $$
    P(B_1\mid B_2) = \frac{\frac{b(b+d)}{(r+b)(r+b+d)}}{\frac{b}{r+b}} = \frac{b+d}{r+b+d}.
    $$
\end{enumerate}
\end{proof}





\begin{framed}
\begin{exercise}
考虑Polya坛子模型. 设\(B_n\)表示第\(n\)个球是蓝色的事件. 证明\(P(B_n) = P(B_1), \forall n \geq 1\).
\end{exercise}
\end{framed}


\begin{proof}
用数学归纳法完成证明. 已经证明\(P(B_1) = P(B_2) = \frac{b}{r+b}\).
记第\(n\)次取球时坛子中球的总个数为\(T_n\), 显然\(T_n = r+b + (n-1)d\).
若对\(n = k\)时成立\(P(B_k) = \frac{b}{r+b}\), 则当\(n = k+1\)时, 
$$
\begin{aligned}
    P(B_{k+1}) &= P(B_k) P(B_{k+1} \mid B_k) + P(B_k^c) P(B_{k+1} \mid B_k^c) \\
    &= \frac{b}{r+b} \frac{P(B_k) T_k+d}{T_k+d} + \frac{r}{r+b} \frac{P(B_k)T_k}{T_k+d} \\
    & = \frac{\frac{b^2T_k}{r+b} + bd + \frac{rbT_k}{r+b}}{(r+b)(T_k+d)} \\
    &= \frac{b^2 T_k +rbd + b^2d + rbT_k}{(r+b)^2(T_k+d)} \\
    &= \frac{b(r+b)(T_k+d)}{(r+d)^2(T_k +d)} \\ & = \frac{b}{r+b} = P(B_1).
\end{aligned}
$$
于是证明了\(P(B_n) = P(B_1), n\geq 1\).
\end{proof}


\begin{framed}
\begin{exercise}
    考虑Polya坛子模型. 求出在接下来抽出\(n\)个球是蓝色的条件下, 第一个球是蓝色的概率. 并求出当\(n\)趋于无穷时的概率极限.
\end{exercise}
\end{framed}

\begin{proof}
先计算\(P(B_1 \mid B_2 \cap \cdots \cap B_{n+1})\). 
根据条件概率的定义, 
$$
P\left(B_1 \mid B_2 \cap \cdots \cap B_{n+1}\right)=\frac{P\left(B_1 \cap B_2 \cap \cdots \cap B_{n+1}\right)}{P\left(B_2 \cap \cdots \cap B_{n+1}\right)}
$$
由于\(P(B_i) > 0, \forall i\), 因此, 
$$
\begin{aligned}
P\left(B_1 \cap \cdots \cap B_{n+1}\right) & =P\left(B_1\right) \cdot P\left(B_2 \mid B_1\right) \cdot P\left(B_3 \mid B_1 \cap B_2\right) \cdots \cdots P\left(B_{n+1} \mid B_1 \cap \cdots \cap B_n\right) \\
& =\frac{b}{b+r} \cdot \frac{b+d}{b+r+d} \cdot \frac{b+2 d}{b+r+2 d} \cdots \cdot \frac{b+n d}{b+r+n d} \\
& =\prod_{k=0}^n \frac{b+k d}{b+r+k d} .
\end{aligned}
$$
对分母, 
$$
\begin{aligned}
    P\left(B_2 \cap \cdots \cap B_{n+1}\right) &= P\left(B_1 \cap B_2 \cap \cdots \cap B_{n+1}\right) + P\left(B_1^c \cap B_2 \cap \cdots \cap B_{n+1}\right) \\
    &= \prod_{k=0}^n \frac{b+k d}{b+r+k d} + \frac{r}{b+r} \cdot \frac{b}{b+r+d} \cdots \frac{b+(n-1)d}{b+r+nd} \\
    &= \prod_{k=0}^n \frac{b+k d}{b+r+k d} + \frac{r}{b+r} \prod_{k=1}^{\infty} \frac{b+(k-1)d}{b+r+kd}, \\
\end{aligned}
$$
因此, 
$$
\begin{aligned}
    P\left(B_1 \mid B_2 \cap \cdots \cap B_{n+1}\right) &= \frac{\prod_{k=0}^n \frac{b+k d}{b+r+k d}}{\prod_{k=0}^n \frac{b+k d}{b+r+k d} + \frac{r}{b+r} \prod_{k=1}^{\infty} \frac{b+(k-1)d}{b+r+kd}} \\
    &= \frac{\frac{b}{b+r}}{\frac{b}{b+r} + \frac{r}{b+r}\cdot \frac{b}{b+nd} } \\
    & = \frac{b(b+nd)}{b(b+nd) +br} \\ &= \frac{b+nd}{b+nd +r}.
\end{aligned}
$$
根据数学分析, 显然, 
$$
\lim_{n\to \infty} P\left(B_1 \mid B_2 \cap \cdots \cap B_{n+1}\right) = 1.
$$
\end{proof}



\begin{framed}
\begin{exercise}
    一个保险公司为等量的男性和女性司机提供保险. 在任意给定的年份, 男性司机发生事故索赔的概率是\(\alpha\), 且与年份独立. 相应的, 女性司机的概率是\(\beta\).
    假设保险公司随机抽取一名司机. 
    \begin{enumerate}[a)]
        \item 被抽中的司机今年索赔的概率是?
        \item 被抽中的司机连续两年索赔的概率是?
    \end{enumerate}
\end{exercise}
\end{framed}

\begin{proof}
\begin{enumerate}[a)]
    \item 设事件\(M\)表示抽中的司机是男性, \(C\)表示司机索赔, 根据题意, 
    $$
    P(M) = P(M^c) = \frac{1}{2}, \quad P(C \mid M) = \alpha, \quad P(C \mid M^c) = \beta.
    $$
    于是, 根据全概公式, 
    $$
    P(C) = P(C \mid M)P(M) + P(C \mid M^c)P(M^c) = \frac{\alpha + \beta}{2}.
    $$
    \item 设事件\(\tilde{C}\)表示司机连续两年索赔, 由于索赔与年份独立, 则\(P(\tilde{C} \mid M) = \alpha^2\), \(P(\tilde{C} \mid M^c) = \beta^2\), 
    于是, 根据全概公式, 
    $$
    P(\tilde{C}) = P(\tilde{C} \mid M)P(M) + P(\tilde{C} \mid M^c)P(M^c) = \frac{\alpha^2 + \beta^2}{2}.
    $$
\end{enumerate}
\end{proof}


\begin{framed}
\begin{exercise}
考虑上一题中保险公司的例子. 设\(A_1, A_2\)分别是随机抽中的司机在第一年, 第二年索赔的事件. 证明\(P(A_2|A_1)\geq P(A_1)\).
\end{exercise}
\end{framed}

\begin{proof}
已经有\(P(A_1) = P(A_2) = \frac{\alpha +\beta}{2}, P(A_1 \cap A_2) = \frac{\alpha^2 + \beta^2}{2}\), 
由于\(\alpha^ 2 + \beta^2 \geq 2\alpha \beta, \alpha, \beta \in [0,1]\), 于是\( \frac{\alpha^2 + \beta^2}{\alpha + \beta} \geq \frac{\alpha + \beta}{2} \), 从而
$$
P(A_2 \mid A_1) = \frac{\alpha^2 + \beta^2}{\alpha + \beta} \geq \frac{\alpha + \beta}{2} = P(A_1).
$$
\end{proof}


\begin{framed}
\begin{exercise}
    考虑上一题中保险公司的例子. 求出索赔人是女性的概率.
\end{exercise}
\end{framed}

\begin{proof}
求\(P(M^c \mid C)\), 用Bayes公式, 
$$
P(M^c \mid C) = \frac{P(C\mid M^c)P(M^c)}{P(C)} = \frac{\beta \times \frac{1}{2}}{\frac{\alpha + \beta}{2}} = \frac{\beta}{\alpha + \beta}.
$$
\end{proof}



\begin{framed}
\begin{exercise}
    设\(A_1, A_2, \cdots, A_n\)是独立的事件. 证明它们都不发生的概率小于等于\(\exp\left(-\sum_{i=1}^n P(A_i)\right)\).
\end{exercise}
\end{framed}

\begin{proof}
事件\(A_1, \cdots, A_n\)都不发生, 是指事件\(A_1\)或事件\(A_2\)或...或事件\(A_n\)不发生, 因此考虑
$$
P\left((\cup_{i = 1}^n A_i)^c\right) = P\left(\cap_{i = 1}^n A_i^c\right) \overset{\text{(独立性)}}{=} \prod_{i = 1}^n P(A_i^c) = \exp\left(\sum_{i = 1}^n \log(1 - P(A_i))\right) \leq \exp\left(-\sum_{i = 1}^n P(A_i)\right),
$$
其中, 最后的不等号是因为\(\log(1 - x) \leq -x\), \(\forall x \in [0, 1]\).
\end{proof}






\begin{framed}
\begin{exercise}
    设\(A, B\)是两事件, \(P(A) > 0\). 证明\(P(A \cap B | A\cup B) \leq P(A\cap B | A)\).
\end{exercise}
\end{framed}


\begin{proof}
    由于\(P(A\cup B) \geq P(A) >0\), 
    由条件概率的定义, 
    \begin{align*}
        P(A \cap B | A\cup B) &= \frac{P(A \cap B \cap (A\cup B))}{P(A\cup B)} = \frac{P(A \cap B)}{P(A\cup B)} \\
        P(A \cap B | A) &= \frac{P(A \cap B)}{P(A)}.
    \end{align*}
    因此, 只需证明\(\frac{P(A \cap B)}{P(A\cup B)} \leq \frac{P(A \cap B)}{P(A)}\), 即\(P(A)P(A\cup B) \leq P(A\cup B)P(A)\), 显然成立.
\end{proof}





\end{document}