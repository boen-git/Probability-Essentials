\documentclass[UTF8, a4paper]{article}
\usepackage{ctex}
\usepackage{graphicx}
\usepackage[margin=2.5cm]{geometry}
\usepackage{subcaption}
\usepackage{amssymb}
\usepackage{amsthm}
\usepackage{amsmath}
\usepackage{enumerate}
\usepackage[backend=bibtex, style=alphabetic]{biblatex}
\usepackage{framed}
\usepackage{mathrsfs} 
\usepackage{xcolor}
\newcommand{\Perp}{\perp\!\!\!\!\perp}
\newtheorem{exercise}{Exercise \#23.}
\newtheorem*{proposition}{命题}
\newtheorem*{remark}{注}
\everymath{\displaystyle}

\addbibresource{my.bib}
\title{Chapter 23: 条件期望}
\author{}
\date{Latest Update: \today}
\begin{document}
\maketitle


对于习题23.1到23.6, 假设\(Y\)是\((\Omega, \mathcal{A}, P)\)正的或可积随机变量, \(\mathcal{G}\)是\(\mathcal{A}\)的子\(\sigma\)-代数.

\begin{framed}
\begin{exercise}
证明: \(|\mathbb{E}\{Y|\mathcal{G}\}| \leq \mathbb{E}\{|Y| \mid \mathcal{G}\}\).
\end{exercise}
\end{framed}

\begin{proof}
若\(Y \geq 0\), 则\(|\mathbb{E}\{Y|\mathcal{G}\}| = \mathbb{E}\{|Y| \mid \mathcal{G}\}\).

若\(Y\)可积, 根据绝对值函数\(\varphi(x) = |x|\)是凸函数, 则由Jensen不等式(Thm 23.9)立刻得出结论.
\end{proof}


\begin{framed}
\begin{exercise}
假设\(\mathcal{H} \subset \mathcal{G}\), 其中\(\mathcal{H}\)是\(\mathcal{G}\)的子\(\sigma\)-代数. 证明: $$\mathbb{E}\{\mathbb{E}\{Y|\mathcal{G}\}|\mathcal{H}\} = \mathbb{E}\{Y|\mathcal{H}\}.$$
\end{exercise}
\end{framed}

\begin{proof}
由条件: \(\mathcal{H} \subset \mathcal{G}\).
首先, \(\mathbb{E}(Y|\mathcal{G})\)是\(\mathcal{G}\)可测的.
要证对于任意的\(H \in \mathcal{H} \subset \mathcal{G}\), 有
$$
\int_H \mathbb{E}\{\mathbb{E}\{Y|\mathcal{G}\}|\mathcal{H}\} dP = \int_H \mathbb{E}\{Y|\mathcal{H}\} dP \overset{by\,def}{=} \int_H Y dP.
$$
而左边的定义是\(\forall H \in \mathcal{H}\),
$$
\int_H \mathbb{E}\{\mathbb{E}\{Y|\mathcal{G}\}|\mathcal{H}\} dP = \int_H \mathbb{E}\{Y|\mathcal{G}\} dP.
$$
而\(H \in \mathcal{G}\), 
$$
\int_H \mathbb{E}\{Y|\mathcal{G}\} dP = \int_H Y dP.
$$
显然.
\end{proof}


\begin{framed}
\begin{exercise}
证明: \(\mathbb{E}\{Y\mid Y\} = Y\) a.s.
\end{exercise}
\end{framed}


\begin{proof}
由定义, 显然.
\end{proof}



\begin{framed}
\begin{exercise}
证明若\(|Y| \leq c\) a.s. 则\(\mathbb{E}\{Y|\mathcal{G}\} \leq c\) a.s.
\end{exercise}
\end{framed}


\begin{proof}
根据引理23.1, 定理23.4, 显然.
\end{proof}


\begin{framed}
\begin{exercise}
若\(Y = \alpha\) a.s., 其中\(\alpha\)是常数, 证明: \(\mathbb{E}\{Y|\mathcal{G}\} = \alpha\) a.s.
\end{exercise}
\end{framed}

\begin{proof}
根据定义, 只需验证, \(\forall G \in \mathcal{G}\), 
$$
\int_G \mathbb{E}\{Y|\mathcal{G}\} dP = \int_G \alpha dP.
$$
得证.
\end{proof}


\begin{framed}
\begin{exercise}
若\(Y\)是正的, 证明\(\{\mathbb{E}\{Y \mid \mathcal{G}\} = 0\} \subset \{Y = 0\}\), 以及\(\{Y = +\infty\} \subset \{\mathbb{E}\{Y\mid \mathcal{G}\} = + \infty\}\) a.s.
\end{exercise}
\end{framed}

\begin{proof}
若不然, 根据\(Y \geq 0\)
$$
P(Y >0, \mathbb{E}\{Y|\mathcal{G}\} = 0) \neq 0.
$$
而根据概率的连续性, 
$$
\lim_{n\to \infty} P\left(Y \geq \frac{1}{n}, \mathbb{E}\{Y|\mathcal{G} = 0\}\right) = P( Y > 0, \mathbb{E}\{Y|\mathcal{G} = 0\}) \neq 0.
$$
于是, 存在充分大的\(N\)使得当\(n > N\)时, 
$$
P\left(Y \geq \frac{1}{n}, \mathbb{E}\{Y|\mathcal{G} = 0\}\right) > 0.
$$
根据\(\mathbb{E}(Y | \mathcal{G})\)的定义, \(\forall G \in\mathcal{G}\)-meas, 
$$
\langle G, \mathbb{E}\{Y|\mathcal{G}\}\rangle = \langle G, Y\rangle.
$$
取\(G = \mathbb{I}\{\mathbb{E}(Y|\mathcal{G}) = 0\}\), 则
$$
\langle G, \mathbb{E}\{Y|\mathcal{G}\}\rangle = 0, \quad \text{(根据运算)}
$$
对于\(\langle G, Y \rangle, \forall n > 0\), 
$$
0 = \langle \mathbb{I}\{\mathbb{E}(Y|\mathcal{G}) = 0\}, Y \rangle \geq \langle \mathbb{I}\{\mathbb{E}(Y|\mathcal{G}) = 0\}, Y \mathbb{I}_{\{Y \geq 1/n\}} \rangle \geq \frac{1}{n} P(Y \geq 1/n, \mathbb{E}(Y|\mathcal{G}) = 0).
$$
矛盾! 所以\(\{\mathbb{E}\{Y \mid \mathcal{G}\} = 0\} \subset \{Y = 0\}\) a.s.


同理, 要证 
$$
\{Y = \infty\} \subset \{\mathbb{E}\{Y|\mathcal{G}\} = + \infty\} a.s.
$$
反证法, 若不然, 
$$
P(Y = \infty, \mathbb{E}\{Y|\mathcal{G}\} <\infty) > 0.
$$
于是存在充分大的\(N\)使得
$$
P(Y = \infty, \mathbb{E}\{Y|\mathcal{G}\} \leq N) > 0.
$$
根据\(\mathbb{E}\{Y|\mathcal{G}\}\)的定义, \(\forall G \in \mathcal{G}\),
$$
\int_G \mathbb{E}\{Y|\mathcal{G}\} dP = \int_G Y dP.
$$
取\(G = \{\mathbb{E}\{Y|\mathcal{G}\} \leq N\}\), 则
$$
LHS = \int_{\{\mathbb{E}\{Y|\mathcal{G}\} \leq N\}} \mathbb{E}\{Y|\mathcal{G}\} dP \leq N P(\mathbb{E}\{Y|\mathcal{G}\} \leq N) < \infty.
$$
但是, 
$$
RHS = \int_{\{\mathbb{E}\{Y|\mathcal{G}\} \leq N\}} Y dP \geq \int_{\{Y = \infty, \mathbb{E}\{Y|\mathcal{G} \leq N\}\}} Y dP = \infty.
$$
矛盾! 所以\(\{Y = \infty\} \subset \{\mathbb{E}\{Y|\mathcal{G}\} = + \infty\} a.s.\).


\end{proof}


\begin{framed}
\begin{exercise}
设\(X, Y\)是独立的, 设\(f\)是Borel的使得\(f(X,Y) \in L^1(\Omega, \mathcal{A}, P)\).
令
$$g(x)=\left\{\begin{array}{cc}\mathbb{E}\{f(x,Y)\}&\text{ if }|\mathbb{E}\{f(x,Y)\}|<\infty,\\0&\text{ otherwise.}\end{array}\right.$$
证明: \(g(X)\)是\(\mathbb{R}^1\)上的Borel函数满足
$$
\mathbb{E}\{f(X, Y) \mid X\} = g(X).
$$
\end{exercise}
\end{framed}

\begin{proof}
注意到, 
$$
g^{-1}(B) = \begin{cases}
\{x \in \mathbb{R}: \mathbb{E}\{f(x, Y)\} \in B\}, & \text{ 若 } 0 \not\in B, \\
\{x \in \mathbb{R}: \mathbb{E}\{f(x, Y)\} =0\} \cup \{x \in \mathbb{R}: \mathbb{E}\{f(x, Y)\} = \infty\}, & \text{ 若 } 0 \in B.
\end{cases}
$$
只需证\(\mathbb{E}\{f(x, Y)\}\)是关于\(x\)的Borel函数即可.
事实上, 根据截口函数的可测性, 
$
f(x,Y)
$
是可积的, 再根据Fubini定理, 
$$
\mathbb{E}\{f(x,Y)\} = \int_{\mathbb{R}} f(x,y) dP_Y(y), 
$$
是Borel可测的.
从而\(g(X)\)是Borel函数.

其次, 由于\(X, Y\)是独立的, 对于\(\forall A\in \sigma(X)\), 
$$
\iint_A f(x,y) dP^{(X,Y)} = \int_A \int_{\mathbb{R}} f(x,y) dP_Y(y) dP_X(x) = \int_A \mathbb{E}\{f(x,Y)\} dP_X(x).
$$
由于\(f(X,Y) \in L^1\), 
$$
\int_A g(x) dP_X(x) = \int_A \mathbb{E}\{f(x,Y)\} dP_X(x) = \iint_A f(x,y) dP^{(X,Y)}.
$$
于是, 
$$
\mathbb{E}\{f(X,Y) \mid X\} = g(X).
$$

\end{proof}


\begin{framed}
\begin{exercise}
设\(Y\)是\(L^2(\Omega, \mathcal{A}, P)\)上的随机变量, 假设\(\mathbb{E}\{Y^2 \mid X\} = X^2\)以及\(\mathbb{E}\{Y \mid X\} = X\).
证明\(Y = X\) a.s.
\end{exercise}
\end{framed}

\begin{proof}
考察\(\mathbb{E}\{(Y-X)^2\}\), 
由于\(\forall Z \in L^2(\Omega, \sigma(X), P)\), 
$$
\mathbb{E}\{Z(Y-X)\} = \mathbb{E}\{Z\mathbb{E}\{Y - \mathbb{E}(Y|X)\}\} = 0.
$$
以及
$$
\mathbb{E}\{Z(Y^2-X^2)\} = \mathbb{E}\{Z\mathbb{E}\{Y^2 - \mathbb{E}(Y^2|X)\}\} = 0.
$$
于是, 在第一个式子取\(Z = X\), 在第二个式子取\(Z = 1\),
可得
$$
\mathbb{E}\{(X-Y)^2\} = 0.
$$
从而\(Y = X\) a.s.
\end{proof}



\begin{framed}
\begin{exercise}
设\(Y\)是指数分布随机变量满足\(P(Y > t) = e^{-t}, t >0\).
计算\(\mathbb{E}\{Y \mid Y \wedge t\}\), 其中\(Y \wedge t = \min(t,Y)\).
\end{exercise}
\end{framed}


\begin{proof}
$$
\mathbb{E}\{Y|Y \wedge t = y\} = 
\begin{cases}
    y, & \text{ if } y < t, \\
    t+1, & \text{ if } y \geq t.
\end{cases}
$$
这是因为
$$
\mathbb{E}\{Y|Y>t\} = \frac{\mathbb{E}\{Y\cdot \mathbb{I}(Y>t)\}}{P(Y > t)} = \frac{\mathbb{E}\{Y\cdot \mathbb{I}(Y>t)\}}{e^{-t}} = t+1.
$$
于是 
$$
\mathbb{E}\{Y|Y \wedge t\} = Y \mathbb{I}(Y < t) + (t+1) \mathbb{I}(Y \geq t).
$$
\end{proof}


\begin{framed}
\begin{exercise}[Chebyshev不等式]
证明对于\(X \in L^2\), \(a > 0, P(|X| \geq a \mid \mathcal{G}) \leq \frac{\mathbb{E}\{X^2 \mid \mathcal{G}\}}{a^2}\).
其中\(P(A\mid \mathcal{G}) = \mathbb{E}\{\mathbb{I}_A \mid \mathcal{G}\}\).
\end{exercise}
\end{framed}

\begin{proof}
只需根据\(L^2\)情况下条件期望的单调性和线性性质:
$$
\mathbb{E}\{X^2 \mid \mathcal{G}\} \geq a^2 \mathbb{E}\{\mathbb{I}_{|X| \geq a} \mid \mathcal{G}\}.
$$
\end{proof}



\begin{framed}
\begin{exercise}[Cauchy-Schwarz不等式]
对于\(X,Y \in L^2\), 证明: 
$$(\mathbb{E}\{XY|\mathcal{G}\})^2\leq \mathbb{E}\{X^2|\mathcal{G}\}\mathbb{E}\{Y^2|\mathcal{G}\}.$$
\end{exercise}
\end{framed}

\begin{proof}
取\(Z = X - \lambda Y\), 其中\(\lambda \in \mathbb{R}, Z^2 \geq 0 \),
再根据\(\Delta \leq 0\)和线性性即可.
\end{proof}



\begin{framed}
\begin{exercise}
    设\(X \in L^2\). 证明 $$\mathbb{E}\{(X-\mathbb{E}\{X|\mathcal{G}\})^2\}\leq \mathbb{E}\{(X-\mathbb{E}\{X\})^2\}.$$
\end{exercise}
\end{framed}

\begin{proof}
对右边中间加一项减一项\(\mathbb{E}\{X | \mathcal{G}\}\).
\end{proof}





\begin{framed}
\begin{exercise}
设\(p \geq 1, r \geq p\).
证明: 对于关于一个概率测度的期望来说\(L^p \supset L^r\). 
\end{exercise}
\end{framed}

\begin{proof}
根据Liapunov不等式, 显然.
\end{proof}



\begin{framed}
\begin{exercise}
设\(Z\)是定义在\((\Omega, \mathcal{F}, P)\)上的随机变量, 其中\(Z \geq 0\), \(\mathbb{E}Z = 1\).
定义一个新的概率测度\(Q\), 满足\(Q(\Lambda) = \mathbb{E}\{\mathbb{I}_{\Lambda} Z\}\).
设\(\mathcal{G}\)是\(\mathcal{F}\)的一个子\(\sigma\)-代数, 设\(U = \mathbb{E}\{Z\mid \mathcal{G}\}\).
证明: 
$$
\mathbb{E}_Q\{X \mid \mathcal{G}\} = \frac{\mathbb{E}\{XZ\mid \mathcal{G}\}}{U}, 
$$
其中\(X\)是任意的有界\(\mathcal{F}\)-可测随机变量.
这里\(\mathbb{E}_Q\{X\mid \mathcal{G}\}\)表示随机变量\(X\)关于概率测度\(Q\)的条件期望.
\end{exercise}
\end{framed}

\begin{proof}
    \(\forall G \in \mathcal{G}\), 
一方面, 
$$
\int_G X dQ = \int_G XZ dP = \int_G \mathbb{E}\{XZ\mid \mathcal{G}\} dP .
$$
另一方面, 
$$
\int_G \frac{\mathbb{E}\{XZ|\mathcal{G}\}}{\mathbb{E}\{Z|\mathcal{G}\}} dQ = \int_G \frac{\mathbb{E}\{XZ|\mathcal{G}\}}{\mathbb{E}\{Z|\mathcal{G}\}} Z dP = \int_G \mathbb{E}\{XZ\mid \mathcal{G}\} dP.
$$
\end{proof}




\begin{framed}
\begin{exercise}
证明: 赋范线性空间\(L^p(\Omega, \mathcal{F}, P)\)是完备的, 其中\(1\leq p < \infty\).
\end{exercise}
\end{framed}

\begin{proof}
要证明\(L^p\)是完备的, 只需证明\(L^p\)中的柯西列收敛.
设\(\{X_n\}_{n\geq 1}\)是\(L^p\)中的柯西列, 则对于\(\forall \varepsilon > 0\), 存在\(N\)使得当\(m,n > N\)时,
$$
\|X_n - X_m\|_p < \varepsilon.
$$
特别地, 取\(\varepsilon_k = \frac{1}{2^k}\), 
则存在子列\(\{x_{x_k}\}\)满足
$$
\|x_{n_k} - x_{n_{k+1}}\|_p < \frac{1}{2^k}.
$$

定义\(Y_n =\sum_{k=1}^{n} |x_{n_k} - x_{n_{k+1}}|\), 
则根据Minkowski不等式,
$$
\|Y_n\|_p \leq \sum_{k=1}^{n} \|x_{n_k} - x_{n_{k+1}}\|_p < \sum_{k=1}^{n} \frac{1}{2^k} = 1.
$$
且\(\{Y_n\}\)是单调递增的, 从而存在极限\(Y = \lim_{n\to \infty} Y_n\).
此时
$$
\|Y\|_p = \lim_{n\to \infty} \|Y_n\|_p \leq 1. 
$$
即\(Y \in L^p\), 且\(Y < \infty\) a.s.
于是子列\(\{X_{n_m} = X_{n_1} +\sum_{k=1}^{m} (X_{n_{k+1}} - X_{n_k})\}\)收敛到\(X\) a.s.
最后, 对于一般的\(X_n\)
$$
\left\|X_n-X \right\|_p \leqslant \|X_n-X_{n_k}\|_p+\| X_{n_k}-X_n \|\rightarrow 0 .
$$
\end{proof}



\begin{framed}
\begin{exercise}
设\(X \in L^1(\Omega, \mathcal{F}, P)\)以及\(\mathcal{G}, \mathcal{H}\)是\(\mathcal{F}\)的子\(\sigma\)-代数.
进一步假设\(\mathcal{H}\)独立于\(\sigma(\sigma(X), \mathcal{G})\).
证明: \(\mathbb{E}\{X\mid \mathcal{G}\} = \mathbb{E}\{X\mid \sigma(\mathcal{G}, \mathcal{H})\}\) a.s.
\end{exercise}
\end{framed}

\begin{proof}
Recall that a $\mathcal{E}$-measurable $L^2$-integrable random variable $Y$ equals $\mathbb{E}(X \mid \mathcal{E})$ if, and only if,

$$
\int_D X d \mathbb{P}=\int_D Y d \mathbb{P}
$$

for all $D \in \mathcal{D}$ where $\mathcal{D}$ is a $\cap$-stable generator of the sub-algebra $\mathcal{E}$.(根据单调类定理)

$$
\mathcal{E}:=\sigma(\mathcal{G}, \mathcal{H}) \quad \mathcal{D}:=\{G \cap H ; G \in \mathcal{G}, H \in \mathcal{H}\}
$$


First of all, we note that $\mathcal{D}$ is in fact a $\cap$-stable generator of $\mathcal{E}=\sigma(\mathcal{G}, \mathcal{H})$. Moreover, we have due to the independence of $\mathcal{H}$ and $\sigma(\sigma(X), \mathcal{G})$ :

$$
\begin{aligned}
\int_{G \cap H} \mathbb{E}(X \mid \mathcal{G}) d \mathbb{P} & =\int 1_H \underbrace{\mathbb{E}\left(1_G X \mid \mathcal{G}\right)}_{\sigma(\sigma(X), \mathcal{G})-\text { measurable }} d \mathbb{P} \\
& =\int 1_H d \mathbb{P} \cdot \int 1_G \mathbb{E}(X \mid \mathcal{G}) d \mathbb{P} \\
& =\int 1_H d \mathbb{P} \int_G X d \mathbb{P} \\
% & =\int 1_H d \mathbb{P} \int_G X d \mathbb{P} \\
& =\int 1_G 1_H X d \mathbb{P} \\
& =\int 1_{G \cap H} \mathbb{E}\{X|\mathcal{E}\} d \mathbb{P}
\end{aligned}
$$


% This shows that (1) holds for any $D=G \cap H \in \mathcal{D}$. Consequently, the claim follows.
\end{proof}


\begin{framed}
\begin{exercise}
设\(\{X_n\}_{n\geq 1}\)独立同分布, 且是\(L^1\)上的随机变量.
令\(S_n = X_1 + \cdots + X_n\), \(\mathcal{G}_n = \sigma(S_n, S_{n+1}, ...)\).
证明: \(\mathbb{E}\{X_1 \mid \mathcal{G}_n\} = \mathbb{E}\{X_1 \mid S_n\}\), \(\mathbb{E}\{X_j \mid \mathcal{G}_n\} = \mathbb{E}\{X_j \mid S_n\}, 1 \leq j \leq n\).
证明: \(\mathbb{E}\{X_j \mid \mathcal{G}_n\} = \mathbb{E}\{X_1 \mid S_n\}, 1 \leq j \leq n\).
\end{exercise}
\end{framed}
\begin{remark}
用习题23.16的结论.
\end{remark}

\begin{proof}
注意到\(\mathcal{G}_n = \sigma(S_n, X_{n+1}, X_{n+2}, ...)\).
且\(X_k \Perp \sigma(S_n, X_j), k > n, j \leq n\), 
则根据习题23.16的结论,
\(\mathbb{E}\{X_1 \mid \mathcal{G}_n\} = \mathbb{E}\{X_1 \mid S_n\}\), \(\mathbb{E}\{X_j \mid \mathcal{G}_n\} = \mathbb{E}\{X_j \mid S_n\}, 1 \leq j \leq n\).

最后, 由于同分布, 从而\(\forall A \in \sigma(S_n)\), 
$$
\int_A X_j dP = \int_A X_1 dP, 1 \leq j \leq n.
$$
从而\(\mathbb{E}\{X_j \mid S_n\} = \mathbb{E}\{X_1 \mid S_n\}, 1 \leq j \leq n\).
\end{proof}




\begin{framed}
\begin{exercise}
设\(X_1, X_2, ..., X_n\)独立同分布, 且是\(L^1\)上的随机变量.
证明对于每一个\(1 \leq j\leq n\), 有
$$
\mathbb{E}\left\{X_j \mid \sum_{i = 1}^{n}X_i\right\} = \frac{1}{n} \sum_{i = 1}^{n} X_i.
$$
\end{exercise}
\end{framed}

\begin{remark}
用定理23.2的结论, 对称性来源于独立同分布条件.
\end{remark}

\begin{proof}
根据上一题的结论, 
$$
\mathbb{E}\left\{X_1 \mid \sum_{i = 1}^{n}X_i\right\} = ... = \mathbb{E}\left\{X_n \mid \sum_{i = 1}^{n}X_i\right\} ,
$$
且 
$$
\mathbb{E}\left\{\sum_{j=1}^{n}X_j \mid \sum_{i = 1}^{n}X_i\right\} = \sum_{j=1}^{n} X_j.
$$
立刻得出结论.
\end{proof}




% \medskip

% \printbibliography


\end{document}